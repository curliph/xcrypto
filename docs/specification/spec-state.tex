\subsection{\XCID state}
\label{sec:spec:state}

It is important to recognise the overhead, wrt. both 
time  (e.g., due to it needing to be context switched) 
and 
space (e.g., wrt. additional logic),
relating to {\em any} state added to RV32I by \XCID.
However, it also seems reasonable to align the trade-off(s) involved with 
existing, common cases such as ISEs for floating-point arithmetic (namely
the standard 
F~\cite[Section 8]{SCARV:RV:ISA:I:17}
and
D~\cite[Section 9]{SCARV:RV:ISA:I:17}
extensions).  The addition of a dedicated floating-point register file is
rationalised, for example, by
a) clear separation of duty,
b) additional capacity, leading, e.g., to enhanced performance,
   and
c) additional (specialised) functionality;
a similar argument is true of \XCID, but motivated by a different context
(i.e., cryptographic vs. floating-point workloads).

Note that, particularly due to the nature of workloads supported by \XCID,
challenges relating to management of additional state are important.  For
example, attack vectors such as 
LazyFP~\cite{SCARV:StePre:18}, 
which capitalises on a short-cut wrt. the overhead of context switching
x87 state, {\em must} be robustly mitigated.

% =============================================================================

\subsubsection{$\XCR$ register file}
\label{sec:spec:state:xcr}

\XCID 
demands one additional 
$16$-element, $32$-bit register file;
we refer to this as the $\XCR$ register file, distinguishing it from the
general-purpose $\GPR$ register file specified by RV32I.
Note that

\begin{itemize}
\item RV32I specifies~\cite[Section 2.1]{SCARV:RV:ISA:I:17} that
      $
      \GPR[*][0] = 0 ,
      $
      i.e., that the $0$-th $\GPR$ register is fixed to $0$: any read from 
      said register yields $0$.  The analogous special-case is {\em not} 
      true of 
      $
      \XCR[*][0] ,
      $
      the $0$-th $\XCR$ register.
\item The $\XCR$ register file is used {\em exclusively} by \XCID, but, on
      the other hand, \XCID can {\em also} use the $\GPR$ register file in 
      selected cases.  For example, it uses a $\GPR$ register as

      \begin{itemize}
      \item a source, specifically a base address, for memory accesses,
            (e.g., \VERB[RV]{xc.ld.w} and \VERB[RV]{xc.st.w}; see \REFSEC{sec:spec:instruction:xc.ld.w} and \REFSEC{sec:spec:instruction:xc.st.w}),
      \item a destination for comparison operations,
            (e.g., \VERB[RV]{xc.mequ}; see \REFSEC{sec:spec:instruction:xc.mequ}).
      \end{itemize}

      \noindent
      Doing so aligns with \REFSEC{sec:bg:concept}, where the host core is 
      pitched as a control-path for the co-processor: address computation,
      control-flow orchestration, etc. fall under the remit of the former, 
      since they can be supported by RV32I as is.
\end{itemize}    

% =============================================================================

\subsubsection{$\SPR{XCCR}$ and $\SPR{XCSR}$ Control and Status Registers (CSRs)}
\label{sec:spec:state:csr}

\begin{table}[p]
\begin{center}
\begin{tabular}{|r@{\;}l|cc|}
\hline
\multicolumn{2}{|c|}{Name}      & Address         & Access     \\
\hline\hline
\XCID Control Register & (XCCR) & \RADIX{7C0}{16} & read/write \\
\XCID Status  Register & (XCSR) & \RADIX{FC0}{16} & read-only  \\
\hline
\end{tabular}
\end{center}
\caption{A summary of \XCID CSRs, per~\cite[Table 2.1]{SCARV:RV:ISA:II:17}.}
\label{tab:csr}
\end{table}

\begin{figure}[p]
\begin{center}
\begin{bytefield}[bitwidth={1.4em},bitheight={8.0ex},endianness=big]{32}
\bitheader{0-31}               
\\
  \bitbox{30}{\rule{\width}{\height}}
& \bitbox{ 1}{\rotatebox{90}{\tiny $S        $}}
& \bitbox{ 1}{\rotatebox{90}{\tiny $U        $}}
\\
\end{bytefield}
\end{center}
\caption{A diagrammatic description of the $\SPR{XCCR}$ register; note that blanked out bits are reserved.}
\label{fig:xccr}
\end{figure}

\begin{figure}[p]
\begin{center}
\begin{bytefield}[bitwidth={1.4em},bitheight={8.0ex},endianness=big]{32}
\bitheader{0-31}               
\\
  \bitbox{19}{\rule{\width}{\height}}
& \bitbox{ 1}{\rotatebox{90}{\tiny $PACK_{16}$}}
& \bitbox{ 1}{\rotatebox{90}{\tiny $PACK_{ 8}$}}
& \bitbox{ 1}{\rotatebox{90}{\tiny $PACK_{ 4}$}}
& \bitbox{ 1}{\rotatebox{90}{\tiny $PACK_{ 2}$}}
& \bitbox{ 1}{\rotatebox{90}{\tiny $PACK_{ 1}$}}
& \bitbox{ 2}{\rule{\width}{\height}}
& \bitbox{ 1}{\rotatebox{90}{\tiny $AES      $}}
& \bitbox{ 1}{\rotatebox{90}{\tiny $MP       $}}
& \bitbox{ 1}{\rotatebox{90}{\tiny $PACK     $}}
& \bitbox{ 1}{\rotatebox{90}{\tiny $BIT      $}}
& \bitbox{ 1}{\rotatebox{90}{\tiny $MEM      $}}
& \bitbox{ 1}{\rotatebox{90}{\tiny $RND      $}}
\\
\end{bytefield}
\end{center}
\caption{A diagrammatic description of the $\SPR{XCSR}$ register; note that blanked out bits are reserved.}
\label{fig:xcsr}
\end{figure}

\begin{table}[p]
\begin{center}
\begin{tabular}{|lc|l|}
\hline
Field        & Index  & Description                                                                                             \\
\hline\hline
$U         $ & $ 0$   & \XCID is accessible in user       mode (set), or not (clear)                                            \\
$S         $ & $ 1$   & \XCID is accessible in supervisor mode (set), or not (clear)                                            \\
\hline
\end{tabular}
\end{center}
\caption{A tabular     description of the $\SPR{XCCR}$ register.}
\label{tab:xccr}
\end{table}

\begin{table}[p]
\begin{center}
\begin{tabular}{|lc|l|}
\hline
Field        & Index  & Description                                                                                             \\
\hline\hline
$RND       $ & $ 0$   & Is RND  feature class, per \REFSEC{sec:bg:feature},                     supported (set), or not (clear) \\
$MEM       $ & $ 1$   & Is MEM  feature class, per \REFSEC{sec:bg:feature},                     supported (set), or not (clear) \\
$BIT       $ & $ 2$   & Is BIT  feature class, per \REFSEC{sec:bg:feature},                     supported (set), or not (clear) \\
$PACK      $ & $ 3$   & Is PACK feature class, per \REFSEC{sec:bg:feature},                     supported (set), or not (clear) \\
$MP        $ & $ 4$   & Is MP   feature class, per \REFSEC{sec:bg:feature},                     supported (set), or not (clear) \\
$AES       $ & $ 5$   & Is AES  feature class, per \REFSEC{sec:bg:feature},                     supported (set), or not (clear) \\
$PACK_{ 1} $ & $ 8$   & Are packed operations on $\VERB[RV]{n} =  1$ sub-word  of $w = 32$ bits supported (set), or not (clear) \\
$PACK_{ 2} $ & $ 9$   & Are packed operations on $\VERB[RV]{n} =  2$ sub-words of $w = 16$ bits supported (set), or not (clear) \\
$PACK_{ 4} $ & $10$   & Are packed operations on $\VERB[RV]{n} =  4$ sub-words of $w =  8$ bits supported (set), or not (clear) \\
$PACK_{ 8} $ & $11$   & Are packed operations on $\VERB[RV]{n} =  8$ sub-words of $w =  4$ bits supported (set), or not (clear) \\
$PACK_{16} $ & $12$   & Are packed operations on $\VERB[RV]{n} = 16$ sub-words of $w =  2$ bits supported (set), or not (clear) \\
\hline
\end{tabular}
\end{center}
\caption{A tabular     description of the $\SPR{XCSR}$ register.}
\label{tab:xcsr}
\end{table}

\XCID 
demands two additional
non-standard, machine-class CSRs~\cite[Section 2]{SCARV:RV:ISA:II:17}
summarised in \REFTAB{tab:csr}:

\begin{enumerate}
\item The \XCID Control Register (XCCR), denoted $\SPR{XCCR}$,
      is described in \REFFIG{fig:xccr} and \REFTAB{tab:xccr}
      the purpose of this CSR is to 
      control the            configuration of an \XCID implementation.
\item The \XCID Status  Register (XCSR), denoted $\SPR{XCSR}$,
      is described in \REFFIG{fig:xcsr} and \REFTAB{tab:xcsr};
      the purpose of this CSR is to 
      inspect the status and configuration of an \XCID implementation,
      including feature identification
      (cf. the x86 \VERB{cpuid}~\cite[Page 3-190--3-224]{SCARV:X86:2:18}).
      Note that 
      \[
      \SPR{XCSR}[*][PACK] = \SPR{XCSR}[*][PACK_{ 1}] \IOR
                            \SPR{XCSR}[*][PACK_{ 2}] \IOR
                            \SPR{XCSR}[*][PACK_{ 4}] \IOR
                            \SPR{XCSR}[*][PACK_{ 8}] \IOR
                            \SPR{XCSR}[*][PACK_{16}]
      \]
      with the single field retained for ease of inspection.
\end{enumerate}

% =============================================================================

\subsubsection{Application Binary Interface (ABI)}
\label{sec:spec:state:abi}

\begin{itemize}
\item For the purposes of the ABI, {\em all} additional state stemming from 
      \XCID is considered to be callee-save: if function $f$ calls function
      $g$, then, for example, $g$ is deemed responsible for using the stack
      to preserve (resp. restore) any content in the $\XCR$ register file 
      it destroys during execution.
\item The first eight registers, 
      i.e., $\XCR[*][0]$ through $\XCR[*][7]$, 
      are considered to be function arguments (including return values).
      All other registers, 
      i.e., $\XCR[*][8]$ upward,
      are considered to be temporaries.
\item Any $\XCR$ registers pushed to the stack should be stored 
      in ascending order.
      That is, since the stack grows downwards, 
      $\XCR[*][i]$
      should be written to a lower address than
      $\XCR[*][j]$
      if $i < j$; if $i = 0$ and $j = 1$, then storing $\XCR[*][i]$ at the
      address $x$ implies storing $\XCR[*][j]$ at the address $x - 4$ for
      example.
\item Any $\XCR$ registers pushed to the stack should be stored 
      at the end of the stack frame, 
      i.e., {\em after} content associated with RV32I (plus any standard 
      extensions thereof, e.g., floating-point content).
\end{itemize}

% =============================================================================
