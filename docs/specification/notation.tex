% =============================================================================

\subsection{General}

\begin{itemize}

\item $x \ASN     y$
      is used to denote an assignment of the value $y$
      to the variable $x$.
\item $x \RAND    Y$
      is used to denote an assignment of a   value $y$,
      sampled uniformly at from the set $Y$ st. $y \in Y$,
      to the variable $x$.
\item $x \TEST{=} y$
      is used to denote a  test or comparison,
      in this case equality,
      between $x$ and $y$ that yields a Boolean (i.e., \TRUE or \FALSE) result.
\item $x \RANGE   y$
      denotes the range of values between $x$ and $y$ inclusive; this implies
      a need to generate intermediate values, st. the notation is only useful
      iff. doing so is obvious {\em and} unambiguous.

\item  $\CARD{X}$
      denotes the cardinality (or size) of some object $X$.
\item $\INDEX{X}{i}$
      refers to an indexed (or numbered)
      element   within some object     $X$, namely the $i$-th such element.
\item $\FIELD{X}{Y}$
      refers to a  named (or labelled)
      field $Y$ within some object     $X$.
\item $\SCOPE{X}{Y}$
      refers to a  named (or labelled)
      field $Y$ within some name-space $X$;
      this is useful to disambiguate different $Y$ with the same identifier.

\end{itemize}

% -----------------------------------------------------------------------------

\subsection{Numbers}

\begin{itemize}

\item $\RADIX{x}{b}$ denotes $x$ is expressed in radix- or base-$b$; where no
      base is specified, it is safe to assume decimal (i.e., that $b = 10$).
\item $\CARD{\RADIX{x}{b}}$ denotes the number of digits in $x$, if expressed
      in base-$b$.
\item Unless otherwise stated, a little-endian digit ordering is assumed; the
      least-significant (resp. most-significant) digit is thus $\INDEX{x}{0}$
      (resp. $\INDEX{x}{n}$).

\end{itemize}

% -----------------------------------------------------------------------------

\subsection{Bits}

\begin{itemize}

\item The operators $\NOT$, $\AND$, $\IOR$ and $\XOR$ denote Boolean NOT, AND,
      OR and XOR respectively, with $\NAND$, $\NIOR$ and $\NXOR$ denoting NAND,
      NOR and NXOR; all of these may be overloaded to cater for bit-sequences
      rather than simply bits.
\item $\HW( X )$
      denotes the Hamming weight   of      bit-sequence  $X$,         st.
      \[
      \HW( X )    = \sum_{i=0}^{i<n} \INDEX{X}{i}                   .
      \]
\item $\HD( X )$
      denotes the Hamming distance between bit-sequences $X$ and $Y$, st.
      \[
      \HD( X, Y ) = \sum_{i=0}^{i<n} \INDEX{X}{i} \XOR \INDEX{Y}{i} .
      \]
\item Given a bit-sequence
      \[
      X = \LIST{ \INDEX{X}{n-1}, \ldots, \INDEX{X}{1}, \INDEX{X}{0} } 
      \]
      and assuming $w$ divides $\CARD{X}$, $\INDEX[w]{X}{i}$ is used to denote 
      the $i$-th sub-sequence of length $w$ in $X$: for $n = 16$, for example, 
      we have that
      \[
      \begin{array}{lcl}
      \INDEX[4]{X}{0} &=& \LIST{ \INDEX{X}{ 3}, \INDEX{X}{ 2}, \INDEX{X}{ 1}, \INDEX{X}{ 0}                                                             } \\
      \INDEX[4]{X}{2} &=& \LIST{ \INDEX{X}{11}, \INDEX{X}{10}, \INDEX{X}{ 9}, \INDEX{X}{ 8}                                                             } \\
      \INDEX[8]{X}{0} &=& \LIST{ \INDEX{X}{ 7}, \INDEX{X}{ 6}, \INDEX{X}{ 5}, \INDEX{X}{ 4}, \INDEX{X}{ 3}, \INDEX{X}{ 2}, \INDEX{X}{ 1}, \INDEX{X}{ 0} } \\
      \INDEX[8]{X}{1} &=& \LIST{ \INDEX{X}{15}, \INDEX{X}{14}, \INDEX{X}{13}, \INDEX{X}{12}, \INDEX{X}{11}, \INDEX{X}{10}, \INDEX{X}{ 9}, \INDEX{X}{ 8} } \\
      \end{array}
      \]
      for example.  In essence, this is a short-hand st.
      \[
      \INDEX[w]{X}{i} \equiv \LIST{ \INDEX[w]{X}{w \cdot i}, \INDEX[w]{X}{w \cdot i + 1}, \ldots, \INDEX[w]{X}{w \cdot i + w - 1} } .
      \]
\item $\LSB[l]{X}$ (resp. $\MSB[l]{X}$) denotes the $l$ least- (resp. most-)
      significant bits of $X$ (where we assume $l = 1$ if omitted): given a
      bit-sequence
      \[
      X = \LIST{ \INDEX{X}{n-1}, \ldots, \INDEX{X}{1}, \INDEX{X}{0} } ,
      \]
      for example, and assuming $l \leq \CARD{X}$, we have that
      \[
      \begin{array}{lcl}
      \LSB[l]{X} &=& \LIST{ \INDEX{X}{l-1},                 \ldots, \INDEX{X}{  1}, \INDEX{X}{  0} } \\
      \MSB[l]{X} &=& \LIST{ \INDEX{X}{n-1}, \INDEX{X}{n-2}, \ldots                  \INDEX{X}{n-l} } \\
      \end{array}
      \]

\end{itemize}

% -----------------------------------------------------------------------------

\subsection{Operations}

\begin{itemize}

\item The operators $\LSH$ and $\RSH$ denote left- and right-shift; $\LRT$ and
      $\RRT$ denote left- and right-rotate (beware of context: $\ll$ and $\gg$
      are also used to denote ``much less than'' and ``much greater than'').
\item For some $x$, 
      we let
      $\EXT[w]{0  }( x )$
      and
      $\EXT[w]{\pm}( x )$
      respectively denote
      zero- or sign-extension to $w$-bits
      (allowing omission of either specifier where appropriate).
\item For some operator $\odot$, 
      we let
      $\OP[w][s]{\odot}$
      and 
      $\OP[w][u]{\odot}$
      respectively denote 
      $w$-bit signed and unsigned variants
      (allowing omission of either specifier where appropriate).

\end{itemize}

% -----------------------------------------------------------------------------

\subsection{Architecture}

\begin{itemize}

\item $\RNG$
      denotes the Random Number Generator (RNG) object;
      assignments to and from said object are overloaded st. if
      it occurs on the RHS this means sampling               entropy,
      whereas if
      it occurs on the LHS this means injecting (or seeding) entropy.
\item $\MEM$
      denotes the byte-addressable memory;
      $\MEM[*][i]$ 
      denotes the $i$-th,
            $8$-bit 
      entry in said memory.
\item $\GPR$ 
      denotes the 
      general-purpose, RV32I register file;
      $\GPR[*][i]$ 
      denotes the $i$-th,
      $\RVXLEN$-bit
      entry in said register file.
\item $\XCR$ 
      denotes the 
      special-purpose, \XCID register file;
      $\XCR[*][i]$ 
      denotes the $i$-th,
           $32$-bit 
      entry in said register file.

% TODO: XCCR, XCSR

\end{itemize}

% =============================================================================
