% =============================================================================

Versus a general case, cryptographic workloads are challenging in that they
typically
a) require computationally intensive, somewhat niche functionality,
   and
b) form a central target in what is a complex, evolving attack surface.
The former is a particular issue, in the sense cryptography is typically an
enabling technology vs. a feature per se: from the perspective of the user,
it represents pure overhead.  Efficiency is a goal in and of itself then, 
but also as an enabler for security.  One cannot (or at least {\em should} 
not) compromise security to meet efficiency requirements, and so delivering
higher efficiency can be pitched as an enabler for countermeasures against 
attack (since there is more margin within which to do so).

This document acts as the specification for a 
greenfield, non-standard extension~\cite[Section 21.1]{SCARV:RV:ISA:I:17} 
to the RISC-V 
RV32I~\cite[Section 2]{SCARV:RV:ISA:I:17}
base ISA, which we dub \XCID; it forms an output from the SCARV\footnote{
\url{http://www.scarv.org}, \url{http://www.github.com/scarv}
} project, funded by EPSRC\footnote{
\url{http://gow.epsrc.ac.uk/NGBOViewGrant.aspx?GrantRef=EP/R012288/1}
} as part of the UK-based RISE\footnote{
\url{http://www.ukrise.org}
} initiative.
The \XCID ISE aims to support software implementations of both symmetric and 
asymmetric cryptographic primitives.  By analogy, one could view this remit
as similar to that of a floating-point co-processor, but cryptography: 
per the above, it aims to enable
a) efficient
   and
b) secure
execution of said implementations.

Note this document {\em is} a specification for \XCID, but categorically
{\em is not} an implementation guide: we provide a separate document for 
that purpose.  In order to avoid the specification becoming too verbose,
we defer coverage of related work and detailed design notes to
\REFAPPX{appx:related}
and
\REFAPPX{appx:design}
respectively.  It follows a semantic versioning\footnote{
\url{http://semver.org}
} (or major/minor/patch) convention, with a summarised changelog\footnote{
\url{http://keepachangelog.com}
} maintained in
\REFAPPX{appx:changelog}.
In line with the current version, the specification should be viewed as an
initial prototype or draft; statements such as 
``\XCID is   X'' 
or
``\XCID does Y''
should be carefully qualified with {\em currently} vs. {\em definitively}.  
In particular, we expect some degree of iteration and so change to emerge 
from work in progress wrt. design, implementation, and evaluation.

% =============================================================================
