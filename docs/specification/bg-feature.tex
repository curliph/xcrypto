% =============================================================================

\subsection{Feature classification}
\label{sec:bg:feature}

\begin{table}[t]
\begin{center}
\begin{tabular}{|l|cc|}
\hline
Feature class                      & Numeric ID & Mnemonic ID \\
\hline
Core                               & $1$        & C           \\
Scatter/Gather                     & $2.1$      & S           \\
Randomness                         & $2.2$      & R           \\
Bit-oriented            operations & $2.3$      & B           \\
Packed                  operations & $2.4$      & P           \\
Multi-precision integer operations & $2.5$      & M           \\
\hline
\end{tabular}
\end{center}
\caption{A list of feature classes, plus their associated numeric and mnemonic identifiers.}
\label{tab:feature}
\end{table}

As noted in \REFSEC{sec:bg:design}, \XCID is a {\em meta}-extension in the
sense it has 
a) one     class of   core     (or baseline) features,
   which must {\em always} be supported,
   plus
b) several classes of optional               features,
   which can be {\em selectively} supported to suit.

%This is done because while some of the instructions are useful across a
%broad range of cryptographic kernels, some are more specific, and are only
%relevant to certain algorithms or primitives.

%All of the optional features within the ISE can be emulated using the
%base RISC-V ISE and the Crypto ISE.

The RISC-V naming convention~\cite[Section 22]{SCARV:RV:ISA:I:17} for ISEs
uses a string of single-character identifiers to specify features realised
in an implementation.  We adopt a variant of this approach, wherein
\[
\XCID[x/y/z]
\]
denotes a concrete implementation of \XCID which supports feature classes 
$x$, $y$, and $z$: we assign a numeric and a mnemonic identifier to each
feature class (summarised by \REFTAB{tab:feature}), preferring\footnote{
One disadvantage of this preference is the conflict with mnemonic used by
the RISC-V naming convention~\cite[Section 22]{SCARV:RV:ISA:I:17}; based
on the notation, we assume it is reasonably easy to disambiguate.
} use of the latter (since it yields a more descriptive result).  
For example,
\[
\XCID[1/2.1/2.2/2.3/2.4] ~\equiv~ \XCID[C/S/R/B/P]
\]
describes an implementation which supports
a) the core features,
b) features relating to advanced memory access, namely scatter/gather
   style store and load operations,
c) features relating to generation of randomness,
d) features relating to bit-oriented (i.e., Boolean) operations,
   plus
e) features relating to packed (or DSP-like) arithmetic operations:
such an implementation is tailored to suit symmetric cryptography (e.g.,
AES), but {\em not} asymmetric cryptography (e.g., RSA).

%Software can determine which features of the ISE are implemented by
%interrogating the {\tt CISEF} register detailed in section
%\ref{sec:csr-mccr}.

% =============================================================================

\subsubsection{Class-$1$:   core}
\label{sec:bg:feature:1}

% constrained by assumed memory interface

%The base features present in all implementations of the Crypto ISE are:
%\begin{itemize}
%\item The {\em move} instructions listed in section
%    \ref{sec:move-instructions}.
%\item The memory access instructions listed in section
%    \ref{sec:memory-instructions}, {\em excluding all scatter and gather
%    variants}.
%\item All {\em bitwise} instructions listed in section
%    \ref{sec:bitwise-instructions}.
%\item All {\em packed field} instructions listed in section
%    \ref{sec:packed-field-instructions}.
%\end{itemize}
%
%The ISE contains several instructions listed below which allow memory
%accesses to directly address the $\CPR$ register file. This means we do
%not need to first load to the $\GPR$ register file, then move data into
%the $\CPR$s.
%
%All ISE memory access instructions source their base address from the
%$\GPR$ register file, but read and write data in the $\CPR$s. This
%prevents address manipulation needing to be done in the $\CPR$s.
%\designnote{
%The decision to source the base address from the $\GPR$ set rather
%than a $\CPR$ was because all of the RISC-V address calculation
%instructions work perfectly well on the $\GPR$s and there is no sense
%in duplicating that functionality for this ISE. This free's up
%encoding space, and lets the $\CPR$ state be used for the compute
%work it was designed for rather than mundane address calculation.
%While care must be
%taken to avoid hazards between the different register sets, we
%believe this decision makes sense overall.
%}
%
%ISE load and store instructions must obey the all ordering rules and
%behaviour with regard to {\tt fence} instructions set out in the
%RISC-V user and privilidged manuals.
%
%There is no requirement for the ISE memory instructions to share the same
%address space as the standard ISA. One may wish to completely separate
%the physical memory used to store cryptographic secrets from the rest of
%the virtual / logical memory space, and the ISE does not stop implementers
%from doing so. This allows for hardware root-of-trust type devices to keep
%their keys in tamper-proof memory, while still letting the rest of the
%program exist in standard RAM/Flash.
%
%Generally, the memory access instructions are analogous to
%the standard RISC-V variants but with the following differences:
%
%\begin{itemize}
%\item The halfword and byte variants can specify which halfword or byte of
%      the register they are reading or writing.
%\item There is no sign-extending equivalent of the \ASM{LHU} and \ASM{LBU}
%      instructions.
%\item The base address is taken from the normal RISC-V $\GPR$ set, while the
%      target register is always in the $\CPR$ set.
%\end{itemize}
%
%These instructions will raise a {\em load/store address misaligned} exception
%if: the final source address is not aligned to the data type the instruction
%is loading {\em and} the implementation does not support misaligned accesses.
%The general rule here is that the \ASM{L*.cr} instructions should behave in
%the same way as their RISC-V cousins in terms of exception handling,
%interrupt behaviour and atomicity rules.

% =============================================================================

\subsubsection{Class-$2$:   general-purpose}
\label{sec:bg:feature:2}

% -----------------------------------------------------------------------------

\subsubsection{Class-$2.1$: scatter/gather}
\label{sec:bg:feature:2:1}

%The {\tt SCATTER} and {\tt GATHER} byte/halfword instructions described
%in section \ref{sec:scatter-gather} may be omitted to reduce implementation
%complexity.
%
%The inclusion of these instructions in an implementation is indicated by the
%{\tt SG} bit of the \nameref{sec:csr-mccr}.
%
%An implementation must support all variants of {\tt SCATTER} and {\tt GATHER}
%or none.
%
%If {\tt SCATTER} and {\tt GATHER} instructions are not implemented, then their
%opcodes will cause the implementation to raise an {\em Illegal Instruction}
%exception.

%Many algorithms rely on bringing together non-contiguous data from memory,
%operating on it as a unit, then placing the results non-contiguously back
%into memory. Scatter / Gather instructions are variants of normal load/store
%instructions which allow a set of offsets (rather that just one) to be
%specified and the data at each offset loaded/stored into a single word
%as appropriate.

%%%The \ASM{SCATTER} and \ASM{GATHER} instructions are designed to be used as
%%%s-box lookup instructions. The {\tt .b} variants allow four separate lookups
%%%per instruction into a 256 entry LUT stored in memory, while the {\tt .h}
%%%variants allow two lookups into a 16-bit LUT stored in memory. \ASM{GATHER}
%%%will move the looked-up values from memory into a register, while \ASM{SCATTER}
%%%will move bytes of a register word back into the LUT in memory.
%%%\designnote{
%%%    While it is of-course possible to emulate the {\tt .h} variants of
%%%    \ASM{SCATTER} and \ASM{GATHER} with the {\tt.b} variants, we felt that
%%%    the performance and efficency savings to be had from telling the hardware
%%%    explicitly to work on halfwords were worth the extra instructions.
%%%}
%%%
%%%The address $a_x$ of each byte/halfword to be loaded or stored is computed by
%%%adding an offset $x$ from the {\tt offsets} $\CPR$ register, and adding it
%%%to the $base$ $\GPR$ register. For the {\tt .b} variants, there are four
%%%offsets each corresponding to the four bytes of the offsets source register
%%%word. For the {\tt .h} variant, there are two offsets corresponding to the
%%%upper and lower halfwords of the registers.
%%%
%%%Both instructions use the $\GPR$s to source their base address, while the
%%%offset indices into the LUT come from the $\CPR$ registers. This mirrors
%%%how the ISE load and store instructions compute their addresses.
%%%
%%%The scatter instruction requires three register file reads.
%%%Because the reads are split across the $\GPR$ and $\CPR$ registers,
%%%no {\em extra} ports are needed on the RISC-V register file beyond the
%%%standard two-read-one-write set.
%%%
%%%An implementation may perform the memory accesses in any order. The instructions
%%%will raise address misalignment exceptions in line with how misalignment is
%%%handled in the rest of the ISA.
%%%
%%%If any of the memory accesses cause
%%%a {\em load access fault, load page fault, store page fault} or
%%%{\em store access fault} then the address of the access which caused the
%%%exception is written to the {\tt MTVAL} csr. The value taken by the
%%%destination register bytes should an exception occur at any point in the
%%%instruction is {\em implementation dependent}.
%%%
%%%Implementations may abort execution of \ASM{SCATTER} or \ASM{GATHER} part way
%%%though in order to service an interrupt. If the instructions are aborted
%%%early, the value taken by the destination register bytes is {\em
%%%implementation dependent}. The implementation is responsible for ensuring
%%%that the {\tt EPC} register is set appropriately depending on whether the
%%%interrupt is taken during, or at the end of, the instruction.
%%%\designnote{Implementation of these instructions should consider the impact of
%%%shared memory systems, since it will be possible for other agents in the
%%%system to access memory during the execution of this instruction. In systems
%%%possessing a cache, programmers must be mindful that these instructions
%%%can cause odd behaviour due to the non-linear access pattern the instruction
%%%creates.}
%%%
%%%The rationale for these instructions is two fold. They help with code density,
%%%since at-least four instructions worth of work can now be done
%%%in one instruction. They can also help with energy efficiency, since fewer
%%%instructions need to be fetched and travel through an execution pipeline in
%%%order to achieve the same goal. They are not designed to enhance performance
%%%(in terms of cycles to complete an operation), but this may be a side-effect
%%%of some implementations.

% -----------------------------------------------------------------------------

\subsubsection{Class-$2.2$: randomness}
\label{sec:bg:feature:2:2}

%\subsection{Randomness}
%
%Implementations can optionally implement the Crypto ISE randomness interface
%instructions {\tt RSEED.cr} and {\tt RSAMP.cr}.
%
%The inclusion of these instructions in an implementation is indicated by the
%{\tt R} bit of the \nameref{sec:csr-mccr}.
%
%If the random interface instructions are not implemented, then their opcodes
%will cause the implementation to raise an {\em Illegal Instruction}
%exception.

% -----------------------------------------------------------------------------

\subsubsection{Class-$2.3$: bit-oriented            operations}
\label{sec:bg:feature:2:3}

Particularly wrt. symmetric primitives, the use of bit-oriented operations 
is common in implementations of cryptography.  At a high level, and though 
counterexamples do exist 
(see, e.g., IDEA~\cite{SCARV:LaiMas:90}, RC6~\cite{SCARV:RRSY:98}),
this fact can be explained by use of alternative (i.e., not word-oriented
and/or not supported by arithmetic in most ISAs) structures and concepts,
and thus operations, in relevant designs.  Associated design rationale may
include
a) security
   (e.g., use of $\B{F}_{2^8}$ in AES~\cite{SCARV:FIPS:197}),
   or
b) efficiency
   (e.g., use of hardware-friendly bit-permutations in DES~\cite{SCARV:FIPS:46_3})
As a result, numerous proposals for associated ISEs exist; these typically
act to accelerate in software what would otherwise represent a high (even 
prohibitive) overhead.  \XCID adopts a similar approach.

Beyond this general argument, the implementation technique of bit-slicing
offers secondary, albeit related motivation for rich(er) support in \XCID.
As introduced by Biham~\cite{SCARV:Biham:97}, bit-slicing is based on
a) a non-standard {\em representation} of data,
   and
b) a non-standard {\em implementation} of functions, which operate on
   said representations:
it essentially describes a given cryptographic primitive as a  ``software 
circuit'' comprising a sequence of bit-wise instructions (e.g., NOT, AND, 
and OR).  Although not a general-purpose technique, when applicable, use
of bit-slicing can offer advantages that include constant-time execution 
and hence immunity from cache-based side-channel attacks
(see, e.g.,~\cite{SCARV:KasSch:09}).
In the design of Serpent~\cite[Page 232]{SCARV:BihAndKnu:98}, there is a
suggestion for accelerating bit-sliced implementations via a ``BITSLICE 
instruction'' or ISE; the suggestion was later investigated in detail by 
Grabher et al.~\cite{SCARV:GraGroPag:08}.  In both cases, the idea is to 
``compress'' a sub-circuit, i.e., the sequence of bit-wise instructions 
representing an $n$-input Boolean function, into a Look-Up Table (LUT): 
the LUT is first configured with a truth table for the function, and then 
accessed to apply said function.  

% -----------------------------------------------------------------------------

\subsubsection{Class-$2.4$: packed                  operations}
\label{sec:bg:feature:2:4}

%\subsection{Packed Arithmetic Widths}
%
%By default, the packed arithmetic width instructions detailed in
%section \ref{sec:packed-arithmetic-instructions} support operations on
%packed bit-fields of widths $2$, $4$, $8$, $16$ and $32$.
%
%For extremely resource constrained or application specific implementations,
%some unnecessary pack widths (as defined by the implementer) may be omitted.
%
%Supported pack widths are indicated by the {\tt pX} bits of the
%\nameref{sec:csr-mccr}.
%
%Packed arithmetic instructions operating on supported pack widths are
%executed as normal. If a packed arithmetic instruction operating on an
%unsupported pack width is encountered, the implementation should raise an
%{\em Illegal Instruction} exception.

%It is common for some CPUs / DSP extensions to allow packed ALU operations.
%This is essentially using the normal GPRs to do SIMD operations on multiple
%pieces of data which are either a halfword or a byte in size. We include
%variants of such instructions which operate on data widths commonly
%found in cryptographic algorithms, particularly block ciphers.

% case in point, PRESENT; post-quantum schemes mod 2^{16}

% -----------------------------------------------------------------------------

\subsubsection{Class-$2.5$: multi-precision integer operations}
\label{sec:bg:feature:2:5}

%All of the multi-precision arithmetic instructions listed in section
%\ref{sec:multi-precision-instructions} may be optionally included or omitted
%as a group.
%
%Inclusion of the multi-precision arithmetic instructions is indicated by
%the {\tt MP} bit of the  \nameref{sec:csr-mccr}.
%
%If the multi-precision arithmetic instructions are not implemented, then their
%opcodes will cause the implementation to raise an {\em Illegal Instruction}
%exception.

Modular, multi-precision integer arithmetic forms a foundation for various
cryptographic use-cases, but particularly asymmetric primitives; examples 
include
RSA (relying on arithmetic in $\B{Z}_N$ for large $N$),
and
ECC (relying on arithmetic in $\B{F}_p$ for large $p$).
As such, there is a large body of literature on the implementation of said 
arithmetic; this includes support for implementations via ISEs defined for 
RISC-based processors
(see, e.g.,\cite{SCARV:GroKam:03,SCARV:GroKam:04}).

\XCID includes a set of instructions that mirror the low-level steps found
in higher-level algorithms for multi-precision integer arithmetic; in part,
these mirror features in XS1~\cite[Section 18]{SCARV:XS1:09} for what there
is termed ``long arithmetic''.  In either case, a central design principle 
is that each instruction explicitly accepts {\em all} input required: there
is no additional state (e.g., accumulator register) nor status flags (e.g., 
for carry or borrow).  Such an approach is particularly attractive when set
within the context of RV32I, because it avoids inclusion of status flags
and so forces (somewhat high overhead) software-based carry and overflow 
detection.

% =============================================================================

\subsubsection{Class-$3$:   special-purpose}
\label{sec:bg:feature:3}

% =============================================================================
