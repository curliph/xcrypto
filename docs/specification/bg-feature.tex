% =============================================================================

\subsection{Feature overview}
\label{sec:bg:feature}

% -----------------------------------------------------------------------------

\subsubsection{Class-$1$: randomness}
\label{sec:bg:feature:1}

% -----------------------------------------------------------------------------

\subsubsection{Class-$2$: packed                  operations}
\label{sec:bg:feature:2}

% -----------------------------------------------------------------------------

\subsubsection{Class-$3$: bit-wise                operations}
\label{sec:bg:feature:3}

As introduced by Biham~\cite{SCARV:Biham:97}, bit-slicing is based on
a) a non-standard {\em representation} of data,
   and
b) a non-standard {\em implementation} of functions, which operate on
   said representations:
essentially it describes some cryptographic primitive (e.g., a block
cipher) as a  ``software circuit'' comprising a sequence of bit-wise 
instructions (e.g., NOT, AND, and OR).  Although not general-purpose,
bit-slicing offers advantages including constant-time execution and 
hence immunity from cache-based side-channel attacks.

The proposal of Serpent~\cite[Page 232]{SCARV:BihAndKnu:98} includes 
a suggestion for accelerating bit-sliced implementations through use 
of a ``BITSLICE instruction''; this suggestion was later investigated 
in a more concrete sense by Grabher et al.~\cite{SCARV:GraGroPag:08}.
In both, the idea is to ``compress'' a sub-circuit, i.e., a sequence 
of bit-wise instructions representing an $n$-input Boolean function,
into a Look-Up Table (LUT): essentially, the LUT is first configured 
with a truth table for the function, and then accessed to apply said
function.  

% -----------------------------------------------------------------------------

\subsubsection{Class-$4$: multi-precision integer operations}
\label{sec:bg:feature:4}

Multi-precision (modular) integer arithmetic represents the foundation
of many cryptographic use-cases, e.g., asymmetric algorithms such as
RSA (relying on arithmetic in $\B{Z}_N$ for large $N$),
and
ECC (relying on arithmetic in $\B{F}_p$ for large $p$).
As such, there is a large body of literature on the implementation of
said arithmetic; this includes {\em support} for implementations via
ISEs defined for RISC-based processors
(see, e.g.,\cite{SCARV:GroKam:03,SCARV:GroKam:04}).

The following ISE mirrors a set of ``long arithmetic'' instructions in
XS1~\cite[Section 18]{SCARV:XS1:09}, which, in turn, mirror the set of
primitive operations typical of multi-precision integer arithmetic.  
A central design principle is that each instruction explicitly accepts 
all operands required: no additional state, e.g., accumulator register
or status flags (e.g., carry or borrow), is required.  Note that such
an approach is particularly attractive within the context of RV32I, as
a result of it omitting status flags in favour of software-based carry
and overflow detection.

% =============================================================================

% TODO: integrate or cut
%
%\subsection{Packed ALU Operations}
%
%It is common for some CPUs / DSP extensions to allow packed ALU operations.
%This is essentially using the normal GPRs to do SIMD operations on multiple
%pieces of data which are either a halfword or a byte in size. We include
%variants of such instructions which operate on data widths commonly
%found in cryptographic algorithms, particularly block ciphers.
%
%\subsection{Scatter / Gather}
%
%Many algorithms rely on bringing together non-contiguous data from memory,
%operating on it as a unit, then placing the results non-contiguously back
%into memory. Scatter / Gather instructions are variants of normal load/store
%instructions which allow a set of offsets (rather that just one) to be
%specified and the data at each offset loaded/stored into a single word
%as appropriate.
%
%\subsection{Multi-precision Integer Operations}
%
%Public key cryptography often relies on manipulating very large integers.
%This necessitates efficient overflow handling. RISC-V has no support for
%hardware overflow or carry out detection, making some performance critical
%parts of public key algorithms slower than on equivalent architectures
%with such support.
%
%\subsection{Bit-sliced Cryptography}
%
%Bit-slicing is a technique which treats a CPU with an $N$ bit data-path as
%a set of $N$ SIMD lanes. Bit-slicing is an effective countermeasure to some
%side-channel attacks, but can be cumbersome to implement in a standard set of
%RISC instructions. We introduce a small set of instructions to accelerate
%bit-sliced implementations, similar to \cite{grabher2008light}.

%The Crypto ISE is split into a base feature set, and several optional
%features which can be included only if the implementation needs them.
%This is done because while some of the instructions are useful across a
%broad range of cryptographic kernels, some are more specific, and are only
%relevant to certain algorithms or primitives.
%
%Software can determine which features of the ISE are implemented by
%interrogating the {\tt CISEF} register detailed in section
%\ref{sec:csr-mccr}.
%
%All of the optional features within the ISE can be emulated using the
%base RISC-V ISE and the Crypto ISE.
%
%\subsection{Base Features}
%
%The base features present in all implementations of the Crypto ISE are:
%\begin{itemize}
%\item The {\em move} instructions listed in section
%    \ref{sec:move-instructions}.
%\item The memory access instructions listed in section
%    \ref{sec:memory-instructions}, {\em excluding all scatter and gather
%    variants}.
%\item All {\em bitwise} instructions listed in section
%    \ref{sec:bitwise-instructions}.
%\item All {\em packed field} instructions listed in section
%    \ref{sec:packed-field-instructions}.
%\end{itemize}
%
%\subsection{Packed Arithmetic Widths}
%
%By default, the packed arithmetic width instructions detailed in
%section \ref{sec:packed-arithmetic-instructions} support operations on
%packed bit-fields of widths $2$, $4$, $8$, $16$ and $32$.
%
%For extremely resource constrained or application specific implementations,
%some unnecessary pack widths (as defined by the implementer) may be omitted.
%
%Supported pack widths are indicated by the {\tt pX} bits of the
%\nameref{sec:csr-mccr}.
%
%Packed arithmetic instructions operating on supported pack widths are
%executed as normal. If a packed arithmetic instruction operating on an
%unsupported pack width is encountered, the implementation should raise an
%{\em Illegal Instruction} exception.
%
%\subsection{Randomness}
%
%Implementations can optionally implement the Crypto ISE randomness interface
%instructions {\tt RSEED.cr} and {\tt RSAMP.cr}.
%
%The inclusion of these instructions in an implementation is indicated by the
%{\tt R} bit of the \nameref{sec:csr-mccr}.
%
%If the random interface instructions are not implemented, then their opcodes
%will cause the implementation to raise an {\em Illegal Instruction}
%exception.
%
%\subsection{Scatter/Gather}
%
%The {\tt SCATTER} and {\tt GATHER} byte/halfword instructions described
%in section \ref{sec:scatter-gather} may be omitted to reduce implementation
%complexity.
%
%The inclusion of these instructions in an implementation is indicated by the
%{\tt SG} bit of the \nameref{sec:csr-mccr}.
%
%An implementation must support all variants of {\tt SCATTER} and {\tt GATHER}
%or none.
%
%If {\tt SCATTER} and {\tt GATHER} instructions are not implemented, then their
%opcodes will cause the implementation to raise an {\em Illegal Instruction}
%exception.
%
%\subsection{Multi-precision Arithmetic}
%
%All of the multi-precision arithmetic instructions listed in section
%\ref{sec:multi-precision-instructions} may be optionally included or omitted
%as a group.
%
%Inclusion of the multi-precision arithmetic instructions is indicated by
%the {\tt MP} bit of the  \nameref{sec:csr-mccr}.
%
%If the multi-precision arithmetic instructions are not implemented, then their
%opcodes will cause the implementation to raise an {\em Illegal Instruction}
%exception.
%
%\subsection{Example Feature Sets}
%
%This section lists some example combinations of feature sets and their
%potential use cases.
%
%\begin{enumerate}
%\item Base + Randomness + Multi-precision Arithmetic. Such a subset would
%    act as a very efficient accelerator to any device performing public
%    key cryptography.
%\item Base + Scatter/Gather. This subset works as a compact extension to
%    a core making use of block ciphers such as AES.
%\item Base + All Others. A complete implementation works as a general
%    purpose cryptographic instruction set extension. Inclusion of the
%    packed arithmetic instructions makes efficient bit-sliced implementations
%    easier to express.
%\end{enumerate}
%
%It is anticipated that most cores will implement all of the features of the
%Crypto ISE, and that only small resource constrained implementations will
%pick and choose the ones they need.
