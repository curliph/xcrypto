% =============================================================================

\subsection{Discussion}
\label{sec:spec:discussion}

% TODO
%\designnote{
%The decision to source the base address from the $\GPR$ set rather
%than a $\XCR$ was because all of the RISC-V address calculation
%instructions work perfectly well on the $\GPR$s and there is no sense
%in duplicating that functionality for this ISE. This free's up
%encoding space, and lets the $\XCR$ state be used for the compute
%work it was designed for rather than mundane address calculation.
%While care must be
%taken to avoid hazards between the different register sets, we
%believe this decision makes sense overall.
%}

% TODO
%There is no requirement for the ISE memory instructions to share the same
%address space as the standard ISA. One may wish to completely separate
%the physical memory used to store cryptographic secrets from the rest of
%the virtual / logical memory space, and the ISE does not stop implementers
%from doing so. This allows for hardware root-of-trust type devices to keep
%their keys in tamper-proof memory, while still letting the rest of the
%program exist in standard RAM/Flash.

% TODO
%\designnote{The RISC-V version of \ASM{LUI} is both for constant creation
%(in tandem with \ASM{ORI} and loading addresses.
%Since the zeroing of the low 12 bits by \ASM{LUI} is mainly useful 
%for addresses, we remove this functionality in favour of leaving the
%low half of the register unmodified.
%The rationale being that two instructions can still create and 32-bit
%immediate, but that if we know a register is zero to begin
%with, we can then use only one instruction to create masks which fit
%into only a single halfword.}

% TODO
%\designnote{
%    This instruction must be implemented with care, since a naive
%    implementation will end up with a 32-port LUT, with correspondingly
%    poor timing performance.

%\designnote{This decision was made simply to make the encoding space
%easier to manage. Rather than adding another 5-bit immediate field, it was
%more efficient to re-use the 4-bit lut field of the BOP.cr
%instruction.}

%\designnote{
%    This scheme means the function to compute the actual source registers
%    from the minimum 3-bit encoding is simply to append either {\tt 00}
%    or {\tt 01} depending on the even or odd address respectively.
%}

%%%\designnote{
%%%    While it is of-course possible to emulate the {\tt .h} variants of
%%%    \ASM{SCATTER} and \ASM{GATHER} with the {\tt.b} variants, we felt that
%%%    the performance and efficency savings to be had from telling the hardware
%%%    explicitly to work on halfwords were worth the extra instructions.
%%%}
%%%An implementation may perform the memory accesses in any order. The instructions
%%%will raise address misalignment exceptions in line with how misalignment is
%%%handled in the rest of the ISA.
%%%If any of the memory accesses cause
%%%a {\em load access fault, load page fault, store page fault} or
%%%{\em store access fault} then the address of the access which caused the
%%%exception is written to the {\tt MTVAL} csr. The value taken by the
%%%destination register bytes should an exception occur at any point in the
%%%instruction is {\em implementation dependent}.
%%%
%%%Implementations may abort execution of \ASM{SCATTER} or \ASM{GATHER} part way
%%%though in order to service an interrupt. If the instructions are aborted
%%%early, the value taken by the destination register bytes is {\em
%%%implementation dependent}. The implementation is responsible for ensuring
%%%that the {\tt EPC} register is set appropriately depending on whether the
%%%interrupt is taken during, or at the end of, the instruction.
%%%\designnote{Implementation of these instructions should consider the impact of
%%%shared memory systems, since it will be possible for other agents in the
%%%system to access memory during the execution of this instruction. In systems
%%%possessing a cache, programmers must be mindful that these instructions
%%%can cause odd behaviour due to the non-linear access pattern the instruction
%%%creates.}

% =============================================================================
