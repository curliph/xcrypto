% =============================================================================

\subsection*{Processor/co-processor/accelerator design}
\cite{SCARV:Gutmann:00}
\cite{SCARV:WuWeaAus:01}
\cite{SCARV:MTRGS:99}
\cite{SCARV:GGHJPTW:11}
\cite{SCARV:CosLebDev:16}

\subsection*{Cryptography- or security-specific processor design}
\cite{SCARV:FouMoo:05,SCARV:Fournier:07,SCARV:KocSavGro:08,SCARV:TheSisPne:09,SCARV:TilKirSze:10,SCARV:NREAMM:12,SCARV:YumSav:15,SCARV:RagAmbPar:15,SCARV:AweAus:17,SCARV:YHEF:18}

\subsection*{ISEs: general/codesign/security/scheduling and code generation}
\cite{SCARV:Fiskiran:05}
\cite{SCARV:BarGioMar:09}
\cite{SCARV:RegIen:16}
\cite{SCARV:FazLopOli:18}
\cite{SCARV:KLWGSTW:06,SCARV:GIPTV:06}
\cite{SCARV:RCSBKBLI:09}
\cite{SCARV:ManGre:10,SCARV:ManMagGre:10,SCARV:Manley:11}

\subsection*{ISEs for arithmetic: $\B{Z}_{N}$ and general multi-precision}
\cite{SCARV:Gro:02,SCARV:Gro:03,SCARV:GroKam:03:a,SCARV:GAST:05,SCARV:GroTilSze:07}
\subsection*{ISEs for arithmetic: $\B{F}_{2^m}$ and other fields (e.g., for ECC)}
\cite{SCARV:GroKam:03:b,SCARV:FisLee:04,SCARV:GroKumPaa:04,SCARV:KumPaa:04,SCARV:BBGM:08}
\subsection*{ISEs for arithmetic: misc}
\cite{SCARV:GroKam:03,SCARV:GroSav:04,SCARV:VejPagGro:07}

\subsection*{ISEs for symmetric}
\cite{SCARV:BurMcDAus:00,SCARV:MelElb:08,SCARV:MelElb:10}

\subsection*{ISEs for AES: security-focused}
\cite{SCARV:TilGro:07:a}
\subsection*{ISEs for AES: efficiency-focused}
\cite{SCARV:TilGroSze:05,SCARV:TilGro:06,SCARV:APRJ:11}
\subsection*{ISEs for AES: mixed use (e.g., repurposed)}
\cite{SCARV:TilGro:05,SCARV:TilGro:07:b,SCARV:BBGR:09,SCARV:BosOzeSta:11}

\subsection*{ISEs for table look-up and memory access}
\cite{SCARV:FisLee:01,SCARV:FisLee:05:a,SCARV:FisLee:05:b,SCARV:HilYinLee:08}
\subsection*{ISEs for bit-manipulation (inc. permutation)}
\cite{SCARV:ShiLee:00,SCARV:YanLee:00,SCARV:McGLee:01,SCARV:LeeShiYan:01,SCARV:ShiLee:02,SCARV:ShiYanLee:03,SCARV:LSYRR:04,SCARV:Shi:04,MASCAB:LeeYanShi:05,SCARV:HilYinLee:08,SCARV:HilLee:08,SCARV:ShiYanLee:08,SCARV:Hilewitz:08}
\subsection*{ISEs for bit-slicing}
As introduced by Biham~\cite{SCARV:Biham:97}, bit-slicing is based on
a) a non-standard {\em representation} of data,
   and
b) a non-standard {\em implementation} of functions, which operate on
   said representations:
it essentially describes a given cryptographic primitive as a  ``software 
circuit'' comprising a sequence of bit-wise instructions (e.g., NOT, AND, 
and OR).  Although not a general-purpose technique, when applicable, use
of bit-slicing can offer advantages that include constant-time execution 
and hence immunity from cache-based side-channel attacks
(see, e.g.,~\cite{SCARV:KasSch:09}).
In the design of Serpent~\cite[Page 232]{SCARV:BihAndKnu:98}, there is a
suggestion for accelerating bit-sliced implementations via a ``BITSLICE 
instruction'' or ISE; the suggestion was later investigated in detail by 
Grabher et al.~\cite{SCARV:GraGroPag:08}.  In both cases, the idea is to 
``compress'' a sub-circuit, i.e., the sequence of bit-wise instructions 
representing an $n$-input Boolean function, into a Look-Up Table (LUT): 
the LUT is first configured with a truth table for the function, then 
accessed to apply said function.  

% =============================================================================
