The following tables show the encodings for all instructions in the ISE,
grouped by their layout.

\encodingsel{
    \note{The load/store byte/halfword instructions encode which bytes
    of the destination / source register they are accessing using the
    {\tt ca}, {\tt cc} and {\tt cd} bits.
    For the load instructions, {\tt cd} specifies the byte, and {\tt cc}
    specifies the halfword. For the store instructions {\tt ca}
    specifies the byte, and {\tt cc} also specifies the halfword.
}
}{ fig:encodings-else }


\encodingspx{
Packed register type instruction encodings. 
The fields {\tt ca, cb} and {\tt cc} are used by the {\tt *.px} instructions
to encode the pack width of their operations. The three bits are
concatenated into a 3-bit vector $a,b,c$ and then decoded into pack widths
using the following lookup table. \\
\begin{itemize}
\item $ca,cb,cc = 000 \Rightarrow $ Pack width $1$  $\Rightarrow$ 1 32-bit word
\item $ca,cb,cc = 001 \Rightarrow $ Pack width $2$  $\Rightarrow$ 2 halfwords
\item $ca,cb,cc = 010 \Rightarrow $ Pack width $4$  $\Rightarrow$ 4 bytes
\item $ca,cb,cc = 011 \Rightarrow $ Pack width $8$  $\Rightarrow$ 8 nibbles
\item $ca,cb,cc = 100 \Rightarrow $ Pack width $16$ $\Rightarrow$ 16 crumbs (2-bit values)
\item $ca,cb,cc = 101 $ - Reserved. Should be treated as an invalid opcode.
\item $ca,cb,cc = 110 $ - Reserved. Should be treated as an invalid opcode.
\item $ca,cb,cc = 111 $ - Reserved. Should be treated as an invalid opcode.
\end{itemize}
}{
fig:encodings-rtype
}


\encodingsmp{
Multi-precision integer instruction encodings. All instructions in
this class are identified first by the {\tt f3} field, then by the {\tt c}
field. The {\tt c} field splits comparison operations from compute operations.
The {\tt mo} and {\tt op} fields then further identify the instructions.
Note that the {\tt rd} field for the comparison instructions is a RISC-V
$\GPR$ register address.
}{
fig:encodings-mp
}
