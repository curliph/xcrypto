
The ISE introduces an additional $16$ by $32$-bit register file which
is used exclusively by the instructions in this ISE.
It is to the crypto ISE what the floating-point register file is to the
RISC-V F extension.

Unlike the RISC-V $\GPR$s, the zeroth $\CPR$ is not tied to zero.

The register file addressing is shown in table \ref{tab:state-addr}.

\begin{table}[h!]
\centering
\begin{tabular}{|l|l l l l|}
\hline
\multicolumn{1}{|l|}{3:0} & \textbf{31:24} & \textbf{23:16} & \textbf{15:8} & \textbf{7:0} \\ \hline
{\tt cpr0  } & 3 & 2 & 1 & 0      \\ \hline
{\tt cpr1  } & 3 & 2 & 1 & 0      \\ \hline
{\tt cpr2  } & 3 & 2 & 1 & 0      \\ \hline
{\tt cpr3  } & 3 & 2 & 1 & 0      \\ \hline
{\tt cpr4  } & 3 & 2 & 1 & 0      \\ \hline
{\tt cpr5  } & 3 & 2 & 1 & 0      \\ \hline
{\tt cpr6  } & 3 & 2 & 1 & 0      \\ \hline
{\tt cpr7  } & 3 & 2 & 1 & 0      \\ \hline
{\tt cpr8  } & 3 & 2 & 1 & 0      \\ \hline
{\tt cpr9  } & 3 & 2 & 1 & 0      \\ \hline
{\tt cpr10 } & 3 & 2 & 1 & 0      \\ \hline
{\tt cpr11 } & 3 & 2 & 1 & 0      \\ \hline
{\tt cpr12 } & 3 & 2 & 1 & 0      \\ \hline
{\tt cpr13 } & 3 & 2 & 1 & 0      \\ \hline
{\tt cpr14 } & 3 & 2 & 1 & 0      \\ \hline
{\tt cpr15 } & 3 & 2 & 1 & 0      \\ \hline
\end{tabular}
\caption{This table shows the per-byte layout of the register file.
Note that there is no {\em zero} register in the ISE.
The register with address $0$ will preserve writes.}
\label{tab:state-addr}
\end{table}

\note{
A smart programmer will remember to clear their $\CPR$s of any secret data
before returning execution to some part of the system they did not write.
}

The number of general purpose registers to put into any ISA and ISE is
always contentious.
We chose 16 general purpose registers for this ISE for the following reasons:

\begin{itemize}
\item As demonstrated by ARM, 16 GPRs is plenty for most computation.
Indeed ARM actually has fewer GPRs due to the special nature of some of their
{\em high} registers.
We believe mirroring the base RISC-V ISA and having
32 GPRs would have been overkill for an ISE such as this, even when the
{\tt F} extension adds 32 registers itself.
\item In area-optimised implementations such as micro-controllers, the
total area of a core is often dominated by the register-file.
For embedded applications which do not need all 32 GPRs, one could use the 
{\tt E} extension, and use the area saved by having 16 GPRs to make room
on the die for the crypto ISE.
\item Fewer registers frees up instruction encoding space which would
normally have been needed for register addresses.
\end{itemize}

This is the only extra architectural state required by the ISE.
We followed the RISC-V principles of avoiding implicit or otherwise
hidden state.

\subsection{ABI Standards}

All of the state added in the Crypto ISE is considered {\em callee save}
for the purposes of the ABI.
That is, if {\tt function1} calls 
{\tt function2}, then {\tt function2} is responsible for saving to the
stack any $\CPR$ registers it needs, and popping them off before returning.

The $\CPR$ registers $0..8$ are considered as function arguments
and/or return values. 
All other $\CPR$ registers are considered temporaries.

Any registers pushed to the stack should be written in ascending order.

