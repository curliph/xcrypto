
\subsection{ISE General Purpose Registers}

The ISE introduces an additional $16$ by $32$-bit register file which
is used exclusively by the instructions in this ISE.
It is to the crypto ISE what the floating-point register file is to the
RISC-V F extension.

Unlike the RISC-V $\GPR$s, the zeroth $\CPR$ is not tied to zero.

The register file addressing is shown in table \ref{tab:state-addr}.

\begin{table}[h!]
\centering
\begin{tabular}{|l|l l l l|}
\hline
\multicolumn{1}{|l|}{3:0} & \textbf{31:24} & \textbf{23:16} & \textbf{15:8} & \textbf{7:0} \\ \hline
{\tt c0  } & 3 & 2 & 1 & 0      \\ \hline
{\tt c1  } & 3 & 2 & 1 & 0      \\ \hline
{\tt c2  } & 3 & 2 & 1 & 0      \\ \hline
{\tt c3  } & 3 & 2 & 1 & 0      \\ \hline
{\tt c4  } & 3 & 2 & 1 & 0      \\ \hline
{\tt c5  } & 3 & 2 & 1 & 0      \\ \hline
{\tt c6  } & 3 & 2 & 1 & 0      \\ \hline
{\tt c7  } & 3 & 2 & 1 & 0      \\ \hline
{\tt c8  } & 3 & 2 & 1 & 0      \\ \hline
{\tt c9  } & 3 & 2 & 1 & 0      \\ \hline
{\tt c10 } & 3 & 2 & 1 & 0      \\ \hline
{\tt c11 } & 3 & 2 & 1 & 0      \\ \hline
{\tt c12 } & 3 & 2 & 1 & 0      \\ \hline
{\tt c13 } & 3 & 2 & 1 & 0      \\ \hline
{\tt c14 } & 3 & 2 & 1 & 0      \\ \hline
{\tt c15 } & 3 & 2 & 1 & 0      \\ \hline
\end{tabular}
\caption{This table shows the per-byte layout of the register file.
Note that there is no {\em zero} register in the ISE.
The register with address $0$ will preserve writes.}
\label{tab:state-addr}
\end{table}

\note{
A smart programmer will remember to clear their $\CPR$s of any secret data
before returning execution to some part of the system they did not write.
}

The number of general purpose registers to put into any ISA and ISE is
always contentious.
We chose 16 general purpose registers for this ISE for the following reasons:

\begin{itemize}
\item As demonstrated by ARM, 16 GPRs is plenty for most computation.
Indeed ARM actually has fewer GPRs due to the special nature of some of their
{\em high} registers.
We believe mirroring the base RISC-V ISA and having
32 GPRs would have been overkill for an ISE such as this, even when the
{\tt F} extension adds 32 registers itself.
\item In area-optimised implementations such as micro-controllers, the
total area of a core is often dominated by the register-file.
For embedded applications which do not need all 32 GPRs, one could use the 
{\tt E} extension, and use the area saved by having 16 GPRs to make room
on the die for the crypto ISE.
\item Fewer registers frees up instruction encoding space which would
normally have been needed for register addresses.
\end{itemize}

This is the only extra architectural state required by the ISE.
We followed the RISC-V principles of avoiding implicit or otherwise
hidden state.

\subsection{ISE Control \& Status Registers}

The Crypto ISE adds two Control \& Status Registers (CSRs) to the
standard set. They are used for feature identification and access
control.

\subsubsection{Crypto ISE Feature Register (cisef)}
\label{sec:csr-cisef}

The {\tt cisef} register is a non-standard, read only, M-level CSR
used to identify which parts of the ISE have been implemented.

For more information on the differences between feature sets, see
section \ref{sec:feature-sets}.

\begin{figure}[H]
\centering
\begin{bytefield}[bitwidth=1.6em,endianness=big]{32}
\bitheader{0-31}               \\
\BB{24}{WIRI}
\BB{1}{ R}
\BB{1}{MP}
\BB{1}{SG}
\BB{1}{p32}
\BB{1}{p16}
\BB{1}{ p8}
\BB{1}{ p4}
\BB{1}{ p2}
\end{bytefield}
\captionsetup{singlelinecheck=off}
\caption[x]{\centering Fields of the {\tt cisef} CSR.}
\label{fig:csr-cisef}
\end{figure}

\begin{table}[H]
\centering
\begin{tabular}{l l l}
\toprule
Field& Bits & Description \\\midrule
WIRI & 31..8 &Writes ignored, reads ignored. \\
R    & 7&Indicates if the Random Instructions interface is
         implemented (set) or not (clear). \\
MP   & 6&Indicates if the multi-precision arithmetic instructions
         are implemented (set) or not (clear). \\
SG   & 5&Indicates if the {\tt SCATTER} and {\tt GATHER} instructions
         are implemented (set) or not (clear). \\
p32  & 4&Indicates if pack-widths of 32 are supported (set) or not (clear).\\
p16  & 3&Indicates if pack-widths of 16 are supported (set) or not (clear).\\
p8   & 2&Indicates if pack-widths of  8 are supported (set) or not (clear).\\
p4   & 1&Indicates if pack-widths of  4 are supported (set) or not (clear).\\
p2   & 0&Indicates if pack-widths of  2 are supported (set) or not (clear).\\
\bottomrule
\end{tabular}
\caption{{\tt cisef} register bit field descriptions.}
\end{table}

\subsubsection{Crypto ISE Access Register (cisea)}
\label{sec:csr-cisea}

The {\tt cisea} register is a non-standard, read/write, M-level CSR
used to control which RISC-V privilege levels have access to the
Crypto ISE state and functionality.

\begin{figure}[H]
\centering
\begin{bytefield}[bitwidth=1.6em,endianness=big]{32}
\bitheader{0-31}               \\
\BB{32}{Placeholder}            \\
\end{bytefield}
\captionsetup{singlelinecheck=off}
\caption[x]{\centering Fields of the {\tt cisea} CSR.}
\label{fig:csr-cisea}
\end{figure}

\subsection{ABI Standards}

All of the state added in the Crypto ISE is considered {\em callee save}
for the purposes of the ABI.
That is, if {\tt function1} calls 
{\tt function2}, then {\tt function2} is responsible for saving to the
stack any $\CPR$ registers it needs, and popping them off before returning.

The $\CPR$ registers $0..8$ are considered as function arguments
and/or return values. 
All other $\CPR$ registers are considered temporaries.

Any registers pushed to the stack should be written in ascending order.
That is, as the stack grows downwards, $c1$ should be written (if needed)
to the address following $c0$.

The $\CPR$ registers which are written to the stack during function calls
should be written {\em after} all of the $\GPR$ and floating-point registers
(if the {\tt F} extension is implemented) have been written.
