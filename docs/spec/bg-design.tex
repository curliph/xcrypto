% =============================================================================

\subsection{Design and discussion}
\label{sec:bg:design}

% -----------------------------------------------------------------------------

\subsubsection{Concept}

\begin{itemize}

\item Assuming carte blanche wrt. design and implementation of a processor 
      core, and the wider system it is coupled to, a large design space of 
      candidate approaches exists.  One could, for example, realise a given 
      cryptographic primitive using
      a) hardware-only,
      b) mixed (or hybrid),
         or
      c) software-only
      techniques.
      \ISE focuses on an approach which supports the latter two cases.

\item \ISE is best described as a general-purpose cryptographic ISE, and so
      specified along those lines.  However, the implementation of \ISE is,
      by design, left more open-ended: one could consider options including
      a) a highly   invasive extension to an existing host core,
      b) a somewhat invasive, tightly-coupled co-processor,
         or
      c) a      non-invasive, loosely-coupled co-processor.

\item Although all cited implementation options are viable, in {\em concept}
      we view \ISE as a co-processor technology: wlog. we adopt terminology 
      associated with that option from here on.
      By doing so, we aim to present a clean(er) separation between roles of
      the host core and \ISE co-processor: the latter is tasked with support
      for execution of cryptography (and therefore, in a sense, represents a 
      security boundary), whereas the latter is not.
      Consider, for example and in a general sense, ARM {\sf big.LITTLE}
      (an instance of the general Asymmetric Multi-core Processor (AMP)~\cite{SCARV:Mittal:16} concept),
      which can be viewed as combining and switching between
      use of
      a ``big''    core (which delivers efficiency wrt. instruction throughput)
      and 
      a ``little'' core (which delivers efficiency wrt. energy consumption).
      The underlying goal is to capitalise on the characteristics of each as 
      and when need be, vs. a single core representing a compromise between 
      those characteristics.  Along similar lines, one could imagine some
      form of {\sf efficient.SECURE} analogue: we have a 
      an ``efficient'' core (which ignores   security, focusing on delivery of efficiency)
      and
      a  ``secure''    core (which ignores efficiency, focusing on delivery of   security),
      instantiated by the host core and \ISE co-processor respectively.

\item The following design criteria rationalise this approach, some of which
      are related (or even stem directly from one and other):

      \begin{itemize}
      \item {\bf      Consistency}.
            As far as is reasonable, \ISE should remain consistent with the
            RISC-V philosophy, and associated base ISA; doing so suggests,
            for example, minimising deviation from the existing instruction
            encoding formats and introduction of additional state.

      \item {\bf       Generality}.
            \ISE deliberately avoids inclusion of any functionality-specific 
            features (and therefore associated hardware).  For example, it
            would be trivial to add dedicated support for AES
            (e.g., following~\cite{SCARV:TilGro:06})
            but such support would be AES-specific and is therefore avoided.

      \item {\bf         Security}.
            An unfortunate fact is that security is commonly relegated to a 
            second-class design metric, and so, by implication, viewed as 
            being of secondary importance;
            see, e.g., \cite{SCARV:RKLMR:03,SCARV:RRKH:04,SCARV:BurMutTiw:16}.
            In contrast, \ISE treats security as a first-class metric, and
            so {\em at least} as important as more traditional alternatives; 
            it is reasonable, for example, to trade-off improved security 
            against degradation of instruction throughput.

      \item {\bf    Composability}.
            To mitigate a given attack, a layered approach (cf. defence in 
            depth) is normally preferred: this favours the use of multiple
            countermeasures, vs. a single, perfect panacea.  As such, \ISE
            a) should be viewed as one option for or layer in a solution,
               and
            b) prefers features that can co-exist over those which cannot.

      \item {\bf      Flexibility}. 
            Modulo the status of security as a first-class design metric, a
            goal of \ISE remains to avoid, or at least minimise any ``baked 
            in'' (or hard coded) trade-offs.  Put another way, although the 
            approach used clearly {\em will} imply trade-offs 
            (e.g., vs. efficiency typically delivered by hardware-only techniques), 
            it {\em should} more easily support

            \begin{itemize}
            \item agility wrt. primitive, algorithm, and parameter choices,
                  and
            \item instrumentation of appropriate countermeasures.
            \end{itemize}

      \item {\bf Implementability}. 
            To use useful, it should be possible to implement \ISE with as
            little
            a) overhead   (e.g., wrt. additional logic),
               and
            b) difficulty (e.g., wrt. outright complexity, or complicating factors such as verification)
            as possible.
            This implies a preference for features that avoid a requirement 
            for complex hardware or invasive alteration to the host core.

      \item {\bf    Measurability}.
            Given the remit of extending RV32I, any feature in \ISE should 
            offer reproducible, demonstrable value vs. this baseline; for 
            a given feature, this goal should be supported by provision of 
            associated reference implementations of cryptographically 
            interesting benchmark kernels.

      \end{itemize}

\item All that said, such an approach may naturally be deemed inappropriate 
      for some use-cases; we pitch \ISE as {\em an} approach, not {\em the}
      approach.  In particular, note that many commercialised SoCs 
      (e.g., Google Titan, Rambus CryptoManager) 
      adopt the {\em opposite} approach by preferring to couple a dedicated
      IP core 
      (e.g., a hardware-based AES accelerator) 
      to a RISC-V host core.

\end{itemize}

% -----------------------------------------------------------------------------

\subsubsection{Assumptions}

\begin{itemize}

\item \ISE demands some interaction with an RNG, the concrete instanciation 
      of which is unspecified.
      On one hand, doing so affords flexibility in an implementation; this 
      is important, in that an RNG selection and implementation will likely 
      be technology-specific (e.g., differ for a given FPGA, vs. an ASIC).  
      On the other hand, however, the RNG used is critically important wrt. 
      security: the (difficult) challenge of selecting and implementing 
      a robust RNG instance is assumed.

\item \ISE assumes a memory interface that is shared between the host core 
      and co-processor.  As such, any attack vector
      (see, e.g.,~\cite{SCARV:GYCH:18})
      that (ab)uses the memory interface {\em must} be robustly mitigated, 
      as would be the case without \ISE (i.e., using the host core alone).

\end{itemize}

% -----------------------------------------------------------------------------

\subsubsection{Content}

\begin{itemize}

\item \ISE attempts to follow the RISC-V philosophy of avoiding unnecessary 
      additional state.  Even so, it does introduce

      \begin{itemize}
      \item $1$ additional $16$-entry co-processor register file,
            and
      \item $2$ additional CSRs: one is used to report on features in, and 
            the other to control an implementation of \ISE.
      \end{itemize}

      \noindent
      It is important to recognise 
      the overhead, wrt. both 
      time (e.g., due to it needing to be context switched) 
      and 
      space (e.g., due to the logic required),
      this (or {\em any} additional state) implies.
      However, it seems reasonable to align this overhead with that of any
      extension for floating-point arithmetic (which, in the same way, will
      typically make use of a dedicated floating-point register file: many 
      of the trade-offs involved, e.g., provision of
      a) clear separation of duty,
      b) additional capacity,
         and
      c) additional (specialised) functionality,
      are the same, but motivated by cryptographic workloads.
      Due to the nature of such workloads, however, attack vectors such as
      LazyFP~\cite{SCARV:StePre:18}, which capitalises on a short-cut wrt. 
      the overhead of context switching, {\em must} be robustly mitigated.

\item Lee and Fiskiran introduce (e.g., see~\cite{SCARV:LeeFis:05}) the PLX
      design, which, among other concepts, supports 
      a) sub-word parallelism (cf. packed, or SWAR-like operations),
         and
      b) word size scalability (or data-path scalability).
      First, 
      note that a fully sub-word parallel ISA is orthogonal wrt. all useful
      sub-word sizes; PLX supports $n$-byte sub-word sizes for sane $n$.
      Second,
      note that, in a sense, RISC-V is word size scalable: it can cater for
      implementations that, e.g., have a $32$-bit instructions but a $32$-,
      $64$-, or $128$-bit general-purpose register file (and address space).
      The same concept is useful wrt. the co-processor, and indeed, for the
      multi-precision integer instructions 
      (per \REFSEC{sec:spec:instr:mp})
      in particular.

      The current design considers sub-{\em byte} (e.g., $4$-bit), sub-word
      sizes; doing so is motivated, e.g., by their utility in some classes 
      of light-weight block cipher.  However, it does not currently explore 
      the potential of a scalable word size: it assumes the general-purpose
      and co-processor register file word sizes are {\em both} $32$ bits.

\item Consider a general case, wherein a given ISA has instruction formats
      to allow access to $n$ general-purpose registers, st. 
      $
      n = s + d
      $ 
      for $s$ sources and $d$ destinations, meaning an associated encoding 
      must somehow specify $n$ register addresses.  The special case
      $
      n = 3 = 2 + 1 
      $
      is common, and adopted by RISC-V, but it should nevertheless be clear
      that {\em other} cases can also be useful.  A common ``wrinkle'' in a 
      strict $3$-address case is (full) $( w \times w )$-bit multiplication, 
      which produces a $(2 \cdot w )$-bit product and therefore demands use 
      of {\em two} $w$-bit destination registers.
   
      One of the strategies underlying \ISE is support for larger $s$ and/or 
      $d$.  This is rationalised by a (ideally positive) trade-off between
      a) increased register file complexity, as a result of the requirement
         to support $n$ ports, or, alternatively, multi-cycle operations,
         vs.
      b) increased register file bandwidth.
      The latter enables each instruction to perform more, or richer forms 
      of computation, aligning well with the demands of many cryptographic 
      workloads.  Even {\em if} said trade-off is acceptable, however, it 
      also implies some challenges wrt. instruction encoding.  There seem
      to be several possible options:

      \begin{enumerate}
      \item One could make all  register addresses {\em explicit}.
            For example, XS1 uses a long (i.e., $32$-bit) $6$-address 
            instruction format~\cite[Page 246]{SCARV:XS1:09} 
            to encode
            {\tt lmul}~\cite[Page 146]{SCARV:XS1:09}.
      \item One could make some register addresses {\em implicit}.  
            For example, the x86 $( 32 \times 32 )$-bit multiplication 
            instruction 
            {\tt mull}~\cite[Page 4-144]{SCARV:X86:2a:12} 
            makes implicit uses of {\tt edx} and {\tt edx} as destinations.
      \item One could make some register addresses {\em implied}.
            For example, this approach has been considered within the
            specific context of support for cryptography: 
            Lee and Choi~\cite{SCARV:LeeCho:08} propose Register File
            Extension for Multi-word and Long-word Operation (RFEMLO), 
            where a group of $n$ contiguous registers is identified by 
            one register address plus a group size (or level in their terminology): 
            address $i$ and level $n$ implies use of registers
            \[
            i, i + 1, i + 2, \ldots i + 2^n - 1 .
            \]
            Note that this approach potentially causes an issue wrt.
            registers with specific semantics.  For example, in many
            RISC-like ISAs (including RISC-V), $\GPR[0]$ is fixed to 
            $0$; it may be difficult to include or exclude $\GPR[0]$ 
            in a group as need be.
      \item One could make some register addresses {\em overloaded}.
            For example, ARMv7-A includes a so-called ``unsigned multiply,
            accumulate accumulate'' instruction 
            {\tt umaal}~\cite[Section A8.8.255]{SCARV:ARMv7_M:17} 
            whose format {\em suggests} $n = 4 = 2 + 2$ but in fact 
            reuses the two destination as additional sources.
      \end{enumerate}
      
      \noindent
      Note that several of these approaches have an implication for the
      difficulty of register allocation; the obvious example is that of
      implicit register addresses.  Likewise, there are various generic
      ways to mitigate the encoding pressure (i.e., the availability of 
      at most $w$ bits) given an approach.  For example one could
      
      \begin{enumerate}
      \item restrict access to some subset of the register file 
            (cf. ARM Thumb or RV32E) 
            thereby reducing the number of bits required to encode each  
            register address,
            or
      \item use some form of instruction prefix.
      \end{enumerate}
      
      \noindent
      The current design is somewhat conservative, in the sense it 
      a) supports (upto) $d = 2$ and $s = 3$, 
         optionally using implied specification option for destinations: 
         one register address is explicit and the other can be implied,
      b) can have short register addresses than normal, due to the use
         of a $16$-entry co-processor register file.
      It does not explore the potential for, and trade-offs wrt. larger 
      $d$ or $s$: doing so is demonstrably useful, but, equally implies
      a range of drawbacks (e.g., overhead wrt. co-processor register 
      file size, complexity in instruction en/decoding, requirement for
      multi-cycle instruction execution).

\end{itemize}

% -----------------------------------------------------------------------------

\subsubsection{Compatibility}

\begin{itemize}

\item Per the current list\footnote{
      See, e.g., \url{http://workspace.riscv.org}
      } of (public) RISC-V working groups, 
      \ISE relates to elements of (at least) the following:

      \begin{enumerate}
      \item Cryptographic Extensions Task Group,
      \item BitManip                      Group,
            and
      \item P             Extension  Task Group (i.e., the embedded DSP-like ISE).
      \end{enumerate}

      \noindent
      In some cases overlap exists, potentially representing an opportunity
      for unification; in other cases the underlying goal, approach, and/or
      ethos is distinct.

\item \ISE makes use of the $custom-1$~\cite[Table 19.1]{SCARV:RV:ISA:I:17}
      instruction encoding space 
      (i.e., the $25$-bit space with prefix $\RADIX{0101011}{2}$).

\item The current design is intended to be agnostic wrt. the host core it
      is coupled to, and there is, conceptually, no reason they cannot be 
      used within RV64I and RV128I.
      Likewise, if treated as a prototype, many constituent concepts could
      be mapped onto alternative, brownfield extensions to RV32I.
      As an example, consider the multi-precision integer instructions 
      (per \REFSEC{sec:spec:instr:mp})
      which {\em could} be adopted in RV32I (having been re-specified to
      operate on the general-purpose register file).

\end{itemize}

% =============================================================================
