
This section introduces some of the main concepts and ideas which are
integrated into the ISE.

\subsection{Byte Addressable Register Files}

Most CPU general purpose register files (GPRs) are accessed in machine-word
width chunks. This makes many operations involving sub-word-width data
cumbersome or inefficent.

For example, loading a series of non-contiguous bytes into memory requires
not just loading of the bytes, but shifting and or'ing of the loaded bytes
into the target register.

A byte addressable register file does not have this problem. It can simply
load a data byte directly into the correct part of a register, so long
as the instruction is capable of specifying which byte to put the result in.

\subsection{Packed ALU Operations}

It is common for some CPUs / DSP extensions to allow packed ALU operations.
This is essentialy using the normal GPRs to do SIMD operations on multiple
pieces of data which are either a halfword or a byte in size.

\subsection{Transposed Register Accesses}

In \cite{jung2000register}, the authors describe an alternate addressing mode
for GPRs which allows them to be accessed row wise (as normal) and
column wise. This can potentially speed up operations which rely on both
column and row wise access to some state matrix.

\subsection{Scatter / Gather}

Many algorithms rely on bringing together non-contiguous data from memory,
operating on it as a unit, then placing the results non-contiguously back
into memory. Scater / Gather instructions are varients of normal load/store
instructions which allow a set of offsets (rather that just one) to be
specified and the data at each offset loaded/stored into a single word
as approproate.

