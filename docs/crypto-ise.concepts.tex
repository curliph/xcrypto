
This section introduces some of the main concepts and ideas which are
integrated into the ISE.

\subsection{Packed ALU Operations}

It is common for some CPUs / DSP extensions to allow packed ALU operations.
This is essentially using the normal GPRs to do SIMD operations on multiple
pieces of data which are either a halfword or a byte in size. We include
variants of such instructions which operate on data widths commonly
found in cryptographic algorithms, particularly block ciphers.

\subsection{Scatter / Gather}

Many algorithms rely on bringing together non-contiguous data from memory,
operating on it as a unit, then placing the results non-contiguously back
into memory. Scatter / Gather instructions are variants of normal load/store
instructions which allow a set of offsets (rather that just one) to be
specified and the data at each offset loaded/stored into a single word
as appropriate.

\subsection{Multi-precision Integer Operations}

Public key cryptography often relies on manipulating very large integers.
This necessitates efficient overflow handling. RISC-V has no support for
hardware overflow or carry out detection, making some performance critical
parts of public key algorithms slower than on equivalent architectures
with such support.

\subsection{Bit-sliced Cryptography}

Bit-slicing is a technique which treats a CPU with an $N$ bit data-path as
a set of $N$ SIMD lanes. Bit-slicing is an effective countermeasure to some
side-channel attacks, but can be cumbersome to implement in a standard set of
RISC instructions. We introduce a small set of instructions to accelerate
bit-sliced implementations, similar to \cite{grabher2008light}.
