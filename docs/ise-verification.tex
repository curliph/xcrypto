
Verification of the ISE implementation will be done using a mixture of
constrained random testing and formal methods.
Both methods will use the same single cycle, functional model of the ISE as a
golden reference model (GRM).
Outputs of the implementation will be compared against the GRM for
correctness.

\subsection{Verilog Golden Reference Model}

The GRM will be written using functional verilog.

\begin{itemize}
\item Each instruction will be implemented as a verilog task.
\item The GRM will have the same interface as the implementation.
\item The GRM will snoop on the implementation interface and produce a
    predicted result for all input instructions.
\item All transactions on the memory and CPU/COP interface will be modelled
    and checked against the GRM.
\end{itemize}

Using bounded model checking (BMC) the aim will be to make sure that after
$12$ clock ticks, the model and implementation have not diverged.
After that check is passed, we switch to induction mode and show that the two
will never diverge after that point.

We use $12$ clock ticks because it is enough for all combinations of 
instruction tripples to be executed and experience things like memory
and writeback stalling.

\subsection{Formal Verification Flow}

The formal flow will use Yosys as a model checker to ensure that for any
sequence of inputs, the outputs of the implementation and the GRM will remain
the same.

\subsection{Constrained Random Flow}

The CRT flow will re-use much of the same environment of the formal flow,
but will allow loading of simple assembly program sequences into a
memory model. This will allow us to create a set of fast unit tests, and to
re-create bugs found in the formal flow.

