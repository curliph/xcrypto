

The aim of this project is to
extend the RISC-V RV32I or RV32E base architectures \footnote{
 There is no reason that the ISE would not work with the
 RV64I and RV128I base instruction sets. Though the ISE state
 uses only 32-bit words, and would not take advantage of a
 larger word size without further modification.
} to accelerate common cryptographic
operations associated with block ciphers and public key cryptography.
It proposes a new Instruction Set Extension (ISE)
which acts analogously to a floating-point co-processor,
but targeted at cryptographic operations.
The aim is for the ISE to be applicable to a wide range of algorithms
which work on different bit-widths and with different block sizes.

The key pieces of functionality the ISE aims to provide are:

\begin{itemize}
\item Flexible {\em packed arithmetic}, where multiple values of varying
    lengths can be operated on inside a register in parallel.
\item Memory Scatter/Gather functionality for bytes.
\item Flexible support for intra-word byte, halfword and nibble
      permutation.
\item Multi-precision integer arithmetic.
\end{itemize}

This document {\em is} a specification for an ISE. It {\em is not} an
implementation guide \footnote{In fact, we go out of our way to make sure this
ISE is applicable to as many sizes and styles of implementation as possible.},
nor is it a standalone ISA.
