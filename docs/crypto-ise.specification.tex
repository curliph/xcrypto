
This section specifies the ISE in terms of it's state and the instructions
which operate on that state.

\subsection{Notation}

Lorem ipsum doler set amet.


\subsection{State}

The ISE introduces an additional $16$ by $32$-bit register file. It can
be accessed in {\em standard} mode or {\em transposed} mode.

The addressing is shown diagramatically in table \ref{tab:state-addr}.

Because there are 32 RISC-V $\GPR$s and only 16 ISE $\CPR$s, the spare
register address bit in the ISE instruction encodings is used to determine
whether to access the register file in standard or transpose mode. In this
case, the {\em top} 16 register addresses access transpose mode, while the
{\em bottom} 16 register addresses access standard mode. \designnote{
    This introduces some complexities with regard to forwarding register
    values in a pipelined design, since one must identify hazards on a 
    per-byte basis, rather than on a per-word basis.
}

\begin{table}[h!]
\centering
\begin{tabular}{|l|l l l l|}
\hline
\multicolumn{1}{|l|}{2:0} & \textbf{ } & \textbf{ } & \textbf{ } & \textbf{ } \\ \hline
\textbf{0 }   & r0.3       & r0.2       & r0.1       & r0.0       \\ \hline
\textbf{1 }   & r1.3       & r1.2       & r1.1       & r1.0       \\ \hline
\textbf{2 }   & r2.3       & r2.2       & r2.1       & r2.0       \\ \hline
\textbf{3 }   & r3.3       & r3.2       & r3.1       & r3.0       \\ \hline
\textbf{4 }   & r4.3       & r4.2       & r4.1       & r4.0       \\ \hline
\textbf{5 }   & r5.3       & r5.2       & r5.1       & r5.0       \\ \hline
\textbf{6 }   & r6.3       & r6.2       & r6.1       & r6.0       \\ \hline
\textbf{7 }   & r7.3       & r7.2       & r7.1       & r7.0       \\ \hline
\textbf{8 }   & r8.3       & r8.2       & r8.1       & r8.0       \\ \hline
\textbf{9 }   & r9.3       & r9.2       & r9.1       & r9.0       \\ \hline
\textbf{10}   & r10.3      & r10.2      & r10.1      & r10.0      \\ \hline
\textbf{11}   & r11.3      & r11.2      & r11.1      & r11.0      \\ \hline
\textbf{12}   & r12.3      & r12.2      & r12.1      & r12.0      \\ \hline
\textbf{13}   & r13.3      & r13.2      & r13.1      & r13.0      \\ \hline
\textbf{14}   & r14.3      & r14.2      & r14.1      & r14.0      \\ \hline
\textbf{15}   & r15.3      & r15.2      & r15.1      & r15.0      \\ \hline
\end{tabular}
\quad
\begin{tabular}{|l|l|l|l|l|}
\hline
\multicolumn{1}{|l|}{2:1 \textbackslash 1:0} & \textbf{3} & \textbf{2} & \textbf{1} & \textbf{0} \\ \hline
\multirow{4}{*}{\textbf{0}} & r19.0   & r18.0   & r17.0   & r16.0   \\
                            & r19.1   & r18.1   & r17.1   & r16.1   \\
                            & r19.2   & r18.2   & r17.2   & r16.2   \\
                            & r19.3   & r18.3   & r17.3   & r16.3   \\ \hline
\multirow{4}{*}{\textbf{1}} & r23.0   & r22.0   & r21.0   & r20.0   \\
                            & r23.1   & r22.1   & r21.1   & r20.1   \\
                            & r23.2   & r22.2   & r21.2   & r20.2   \\
                            & r23.3   & r22.3   & r21.3   & r20.3   \\ \hline
\multirow{4}{*}{\textbf{2}} & r27.0   & r26.0   & r25.0   & r24.0   \\
                            & r27.1   & r26.1   & r25.1   & r24.1   \\
                            & r27.2   & r26.2   & r25.2   & r24.2   \\
                            & r27.3   & r26.3   & r25.3   & r24.3   \\ \hline
\multirow{4}{*}{\textbf{3}} & r31.0   & r30.0   & r29.0   & r28.0   \\
                            & r31.1   & r30.1   & r29.1   & r28.1   \\
                            & r31.2   & r30.2   & r29.2   & r28.2   \\
                            & r31.3   & r30.3   & r29.3   & r28.3   \\ \hline
\end{tabular}
\caption{These two table show the per-byte layoute in standard and transposed
addressing modes. On the left is the standard addressing mode, where the
4-bit register address corresponds to rows in the table. On the right
is the transposed mode, where registers are addressed in four rows of
four columns.}
\label{tab:state-addr}
\end{table}

Unlike the RISC-V $\GPR$s, the zeroth $\CPR$ is not tied to zero.

\note{
A smart programmer will remember to clear their $\CPR$s of any secret data
before returning execution to some part of the system they did not write.
}

\subsubsection{ABI Standards}

All of the state added in the Crypto ISE is considered {\em callee save}
for the purposes of the ABI. That is, if {\tt function1} calls 
{\tt function2}, then {\tt function2} is responsible for saving to the
stack any $\CPR$ registers it needs, and poping them off before returning.

Registers which are pushed to the stack in standard or transpose
way must be popped back using the same mode they were stored with.

The $\CPR$ registers $0..4$ are considered as function arguments
and/or return values. Due to the standard / transpose addressing modes, this
means that registers $16..19$  (which alias $0..4$ as per table \ref{tab:state-addr}) are also considered function arguments and/or return values. It is the
programmer's responsibility to correctly interpret results as being in
transposed or standard form.

All other $\CPR$ registers are considered temporaries.

\subsection{Instructions}

All instructions occupy the \encspace encoding space of the RISC-V opcode
map, using the associated \encopcode prefix for bits $6:0$ of each 
instruction.

\subsubsection{Move Instructions}
\paragraph{MVCOP}

Move a $\CPR$ register to a $\GPR$ register.

\begin{itemize}
\item \INST{MVCOP gpr, cpr}{
    $\GPR[*][\OPER{gpr}] \ASN \CPR[*][\OPER{cpr}]$
}
\end{itemize}


\paragraph{MVGPR[.t]}

The complement of \ASM{MVCOP}.
Move a $\GPR$ register to a $\CPR$ register.

\begin{itemize}
\item \INST{MVGPR cpr, gpr}{
    $\CPR[*][\OPER{cpr}] \ASN \GPR[*][\OPER{gpr}]$
}
\end{itemize}

\subsubsection{Packed Arithmetic}
\paragraph{ADD.px}

The instruction \ASM{ADD.px} treats the source and destination registers as
1,2,4 or 8 unsigned, packed fields, and adds corresponding fields together
ignoring any carry bits.

No overflow conditions are tested for, and \ASM{ADD.px} raises no exceptons.

\begin{itemize}
\item \INST{ADD.p8, rd, rs1, rs2}{
$\CPR[*][rd][ 3: 0] \ASN \CPR[*][rs1][ 3: 0] + \CPR[*][rs2][ 3: 0]$\\
$\CPR[*][rd][ 7: 4] \ASN \CPR[*][rs1][ 7: 4] + \CPR[*][rs2][ 7: 4]$\\
$\CPR[*][rd][11: 8] \ASN \CPR[*][rs1][11: 8] + \CPR[*][rs2][11: 8]$\\
$\CPR[*][rd][15:12] \ASN \CPR[*][rs1][15:12] + \CPR[*][rs2][15:12]$\\
$\CPR[*][rd][19:16] \ASN \CPR[*][rs1][19:16] + \CPR[*][rs2][19:16]$\\
$\CPR[*][rd][23:20] \ASN \CPR[*][rs1][23:20] + \CPR[*][rs2][23:20]$\\
$\CPR[*][rd][25:24] \ASN \CPR[*][rs1][25:24] + \CPR[*][rs2][25:24]$\\
$\CPR[*][rd][31:26] \ASN \CPR[*][rs1][31:26] + \CPR[*][rs2][31:26]$\\
}
\item \INST{ADD.p4, rd, rs1, rs2}{
$\CPR[*][rd][ 7: 0] \ASN \CPR[*][rs1][ 7: 0] + \CPR[*][rs2][ 7: 0]$\\
$\CPR[*][rd][15: 8] \ASN \CPR[*][rs1][15: 8] + \CPR[*][rs2][15: 8]$\\
$\CPR[*][rd][23:16] \ASN \CPR[*][rs1][23:16] + \CPR[*][rs2][23:16]$\\
$\CPR[*][rd][31:24] \ASN \CPR[*][rs1][31:24] + \CPR[*][rs2][31:24]$\\
}
\item \INST{ADD.p2, rd, rs1, rs2}{
$\CPR[*][rd][15: 0] \ASN \CPR[*][rs1][15: 0] + \CPR[*][rs2][15: 0]$\\
$\CPR[*][rd][31:16] \ASN \CPR[*][rs1][31:16] + \CPR[*][rs2][31:16]$\\
}
\item \INST{ADD.p1, rd, rs1, rs2}{
$\CPR[*][rd] \ASN \CPR[*][rs1] + \CPR[*][rs2]$\\
}
\end{itemize}

\paragraph{SUB.px}

The instruction \ASM{SUB.px} treats the source and destination registers as
1,2,4 or 8 unsigned, packed fields, and subtracts the {\tt rs2} fields from
the corresponding {\tt rs1} fields.

No overflow conditions are tested for, and \ASM{SUB.px} raises no exceptons.

\paragraph{MUL.px}

The instruction \ASM{MUL.px} treats the source and destination registers as
1,2,4 or 8 unsigned, packed fields, and multiplies the {\tt rs2} fields by
the corresponding {\tt rs1} fields.

No overflow conditions are tested for, and \ASM{MUL.px} raises no exceptons.

\paragraph{SLL.px}

The instruction \ASM{SLL.px} treats the source and destination registers as
1,2,4 or 8 unsigned, packed fields, and shifts the {\tt rs1} fields left
by the value in {\tt SHAMT}.

If the distance to shift by is greater than or equal to the size of each
packed field, then the result of the \ASM{SLL.px} instruction is
{\em implementation dependent}.

\paragraph{SRL.px}

The instruction \ASM{SRL.px} treats the source and destination registers as
1,2,4 or 8 unsigned, packed fields, and shifts the {\tt rs1} fields right
by the value in {\tt SHAMT}.

If the distance to shift by is greater than or equal to the size of each
packed field, then the result of the \ASM{SRL.px} instruction is
{\em implementation dependent}.

\paragraph{ROT.px}

The instruction \ASM{ROT.px} treats the source and destination registers as
1,2,4 or 8 unsigned, packed fields, and rotates the {\tt rs2} fields by
the value in {\tt SHAMT}.

\subsubsection{Bitwise and Twiddle Instructions}

\paragraph{XORT, ANDT, ORT}

The \ASM{XORT}, \ASM{ANDT}, and \ASM{ORT} instructions compute the
appropriate bitwise result of their operation between the {\tt rs1} and
{\tt rs2} operands. The result is then written into {\tt rd}.
\designnote{Originally the bitwise instructions were to have twiddle
functionality built-in, removing the need for a dedicated twiddle instruction.
The encoding space required however did not fit well with the existing RISC-V
register address and function field layout, and was extremely expensive.}

The difference between these instructions and their standard RISC-V
counterparts is that they operate on the $\CPR$ register file, and so
support the transposed addressing modes.

\begin{itemize}
\item \INST{XORT rd, rs1, rs2}{
    $ \CPR[*][\OPER{rd}] \ASN \CPR[*][rs1] \XOR \CPR[*][rs2] $ \\
}
\item \INST{ANDT rd, rs1, rs2}{
    $ \CPR[*][\OPER{rd}] \ASN \CPR[*][rs1] \AND \CPR[*][rs2] $ \\
}
\item \INST{ORT rd, rs1, rs2}{
    $ \CPR[*][\OPER{rd}] \ASN \CPR[*][rs1] \IOR \CPR[*][rs2] $ \\
}
\end{itemize}

Note that by having {\tt rs1} the same as {\tt rs2}, the twiddle
arithmetic instructions can be used to arbitrarily re-arrange bytes
within words. They can also be used to duplicate bytes within words.

\paragraph{TWID}

The \ASM{TWID} instruction is used to {\em twiddle} bytes within words.
It arbitrarily maps the bytes of a source word onto the bytes of a
destination word.

\begin{itemize}
\item \INST{TWID rd, rs1, b0, b1, b2, b3}{
    $ \CPR[*][\OPER{rd}][0] \ASN \CPR[*][\OPER{rs1}][\OPER{b0}] $ \\
    $ \CPR[*][\OPER{rd}][1] \ASN \CPR[*][\OPER{rs1}][\OPER{b1}] $ \\
    $ \CPR[*][\OPER{rd}][2] \ASN \CPR[*][\OPER{rs1}][\OPER{b2}] $ \\
    $ \CPR[*][\OPER{rd}][3] \ASN \CPR[*][\OPER{rs1}][\OPER{b3}] $ 
}

Twiddle can be used to broadcast bytes within words and re-order bytes
across rows and columns via the transpose addressing mode.
\end{itemize}


\subsubsection{Multi-precision Arithmetic}
\paragraph{ADD.m}
\paragraph{SUB.m}
\paragraph{ACC.m}
\paragraph{MAC.m}
\paragraph{SLL.m}
\paragraph{SRL.m}
\paragraph{EQU.m}
\paragraph{LTU.m}
\paragraph{GTU.m}

\subsubsection{Memory Access}

\paragraph{LBU.cr, LHU.cr, LW.cr}

The \ASM{LBU.cr}, \ASM{LHU.cr} and \ASM{LW.cr} instructions are analogous to
the standard RISC-V varients but with the following differences:

\begin{itemize}
\item They support transposition in their destination $\CPR$ register.
\item The halfword and byte varients can specify which halfword or byte of
      the destination register they are targeting.
\item The base address is taken from the normal RISC-V $\GPR$ set, while the
      target register is always in the $\CPR$ set.
\end{itemize}

These instructions will raise a {\em load address misaligned} instruction
if: the final source address is not aligned to the data type the instruction
is loading {\em and} the implementation does not support misaligned accesses.
The general rule here is that the \ASM{L*.cr} instructions should behave in
the same way as their RISC-V cousins in terms of exception handling,
interrupt behaviour and atomicity rules.

\begin{itemize}
\item \INST{LBU.cr  rd, b, imm(rs1)}{
    $addr \ASN  \GPR[*][rs1] + signextend(imm, 32)$ \\
    $\CPR[*][rd][b] \ASN \MEM[*][addr]$
}
\item \INST{LHU.cr  rd, h, imm(rs1)}{
    $addr \ASN  \GPR[*][rs1] + signextend(imm, 32)$ \\
    $b_1 \ASN  (2'00_2 \IOR h) \LSH 1$ \\
    $b_2 \ASN  b_1 \IOR 1'1_2$ \\
    $\CPR[*][rd][b_2] \ASN \MEM[*][addr + 1]$ \\
    $\CPR[*][rd][b_1] \ASN \MEM[*][addr + 0]$
}
\item \INST{LW.cr  rd, imm(rs1)}{
    $addr \ASN  \GPR[*][rs1] + signextend(imm, 32)$ \\
    $\CPR[*][rd][rb + 3] \ASN \MEM[*][addr+3]$ \\
    $\CPR[*][rd][rb + 2] \ASN \MEM[*][addr+2]$ \\
    $\CPR[*][rd][rb + 1] \ASN \MEM[*][addr+1]$ \\
    $\CPR[*][rd][rb + 0] \ASN \MEM[*][addr+0]$
}
\end{itemize}

\note{
    The notation above assumes a byte addressed memory, where {\tt db}
    is a two bit byte address and {\tt dh} is a one bit halfword address.
}

\paragraph{SB.cr, SH.cr, SW.cr}

The \ASM{SB.cr}, \ASM{SH.cr} and \ASM{SW.cr} instructions are analogous to
the standard RISC-V varients but with the following differences:

\begin{itemize}
\item They support transposition in their source data $\CPR$ register.
\item The halfword and byte varients can specify which halfword or byte of
      the source register they store.
\item The base address is taken from the normal RISC-V $\GPR$ set, while the
      source data / destination register is always in the $\CPR$ set.
      \designnote{
        The decision to source the base address from the $\GPR$ set rather
        than a $\CPR$ was because all of the RISC-V address calculation
        instructions work perfectly well on the $\GPR$s and there is no sense
        in duplicating that functionality for this ISE. This free's up
        encoding space, and lets the $\CPR$ state be used for the compute
        work it was designed for rather than mundane address calculation.
        While care must be
        taken to avoid hazards between the difference register sets, we
        belive this decision makes sense overall.
      }
\end{itemize}

These instructions will raise a {\em store address misaligned} exception
if: the final source address is not aligned to the data type the instruction
is storing {\em and} the implementation does not support misaligned accesses.
The general rule here is that the \ASM{L*.cr} instructions should behave in
the same way as their RISC-V cousins in terms of exception handling,
interrupt behaviour and atomicity rules.

\begin{itemize}
\item \INST{SB.cr  rs1, b, imm(rs2)}{
    $addr \ASN  \GPR[*][rs1] + signextend(imm, 32)$ \\
     $\MEM[*][addr] \ASN \CPR[*][rs2][b]$
}
\item \INST{SH.cr  rs1, h, imm(rs2)}{
    $addr \ASN  \GPR[*][rs1] + signextend(imm, 32)$ \\
    $b_1 \ASN  (2'00_2 \IOR h) \LSH 1$ \\
    $b_2 \ASN  b_1 \IOR 1'1_2$ \\
    $\MEM[*][addr + 1] \ASN \CPR[*][rs2][b_2]$ \\
    $\MEM[*][addr + 0] \ASN \CPR[*][rs2][b_1]$
}
\item \INST{SW.cr  rs1, imm(rs2)}{
    $addr \ASN  \GPR[*][rs1] + signextend(imm, 32)$ \\
    $\MEM[*][addr+3] \ASN \CPR[*][rd][rb + 3]$ \\
    $\MEM[*][addr+2] \ASN \CPR[*][rd][rb + 2]$ \\
    $\MEM[*][addr+1] \ASN \CPR[*][rd][rb + 1]$ \\
    $\MEM[*][addr+0] \ASN \CPR[*][rd][rb + 0]$
}
\end{itemize}

\note{
    The notation above assumes a byte addressed memory, where {\tt b}
    is a two bit byte address and {\tt h} is a one bit halfword address.
}

\paragraph{SCATTER, GATHER}

The \ASM{SCATTER} and \ASM{GATHER} instructions are designed to be used as
sbox lookup instructions. They allow four separate lookups per instruction
into a 256 entry LUT stored in memory. \ASM{GATHER} will move the looked-up
values from memory into a register, while \ASM{SCATTER} will move bytes of a
register word back into the LUT in memory.

The address $a_x$ of each byte to be loaded or stored is computed by
adding an offset byte $x$ from the {\tt offsets} $\CPR$ register, and
adding it to the $base$ $\GPR$ register.

Both instructions use the $\GPR$s to source their base address, while the
offset indicies into the LUT come from the $\CPR$ registers. This mirrors
how the ISE load and store instructions compute their addresses.

Both instructions support transpose and standard addressing modes for the
offset and destination registers. The base address register, since it is
a $\GPR$, supports only standard mode addressing, but can be any one of the
$\GPR$s.

\begin{itemize}
\item \INST{GATHER rd, base, offsets}{
    $ a_0 \ASN \GPR[*][base] + \CPR[*][offsets][0]$ \\
    $ a_1 \ASN \GPR[*][base] + \CPR[*][offsets][1]$ \\
    $ a_2 \ASN \GPR[*][base] + \CPR[*][offsets][2]$ \\
    $ a_3 \ASN \GPR[*][base] + \CPR[*][offsets][3]$ \\
    $\CPR[*][rd][0] \ASN \MEM[*][a_0]$ \\
    $\CPR[*][rd][1] \ASN \MEM[*][a_1]$ \\
    $\CPR[*][rd][2] \ASN \MEM[*][a_2]$ \\
    $\CPR[*][rd][3] \ASN \MEM[*][a_3]$
}
\item \INST{SCATTER rs, base, offsets}{
    $ a_0 \ASN \GPR[*][base] + \CPR[*][offsets][0]$ \\
    $ a_1 \ASN \GPR[*][base] + \CPR[*][offsets][1]$ \\
    $ a_2 \ASN \GPR[*][base] + \CPR[*][offsets][2]$ \\
    $ a_3 \ASN \GPR[*][base] + \CPR[*][offsets][3]$ \\
    $ \MEM[*][a_0] \ASN \CPR[*][rs][0]$ \\
    $ \MEM[*][a_1] \ASN \CPR[*][rs][1]$ \\
    $ \MEM[*][a_2] \ASN \CPR[*][rs][2]$ \\
    $ \MEM[*][a_3] \ASN \CPR[*][rs][3]$
}
\end{itemize}

\note{
    The scatter instruction requires three register file reads.
    Because the reads are split across the $\GPR$ and $\CPR$ registers,
    no {\em extra} ports are needed on either register file beyond the
    standard two-read-one-write set.
}

An implementation may perform the memory accesses in any order. The accesses
appear as {\em byte} accesses, so the value of the {\em base} register
need not be word or halfword aligned.

If any of the memory accesses cause
a {\em load access fault, load page fault, store page fault} or
{\em store access fault} then the address of the access which caused the
exception is written to the {\tt MTVAL} csr. The value taken by the
destination register bytes should an exception occur at any point in the
instruction is {\em implementation dependent}.

Implementations may abort execution of \ASM{SCATTER} or \ASM{GATHER} part way
though in order to service an interrupt. If the instructions are aborted
early, the value taken by the destination register bytes is {\em
implementation dependent}. The implementation is responsible for ensuring
that the {\tt EPC} register is set appropriately depending on whether the
interrupt is taken during, or at the end of, the instruction.
\designnote{Implementation of these instructions consider the impact of
shared memory systems, since it will be possible for other agents in the
system to access memory during the execution of this instruction. In systems
posessing a cache, programmers must be mindful that these instructions
can cause odd behaviour due to the non-linear access pattern the instruction
creates. All accesses will be within 256 bytes of one another however,
which may motivate particular alignment of the LUT based on cache-line size}

The rationale for these instructions is two fold. They help with code density,
since atleast four instructions worth of work (eight in RV32) can now be done
in one instruction. They can also help with energy efficiency, since fewer
instructions need to be fetched and travel through an execution pipeline in
order to achieve the same goal. They are not designed to enhance performance
(in terms of cycles to complete an operation), but this may be a side-effect
of some implementations.

\newpage
\subsection{Instruction Listings}

This section lists the assembly notation for the instructions, as well as
their encodings.

{\tt
ADD.p1  rd, rs1, rs2            \\
ADD.p2  rd, rs1, rs2            \\
ADD.p4  rd, rs1, rs2            \\
ADD.p8  rd, rs1, rs2            \\
SUB.p1  rd, rs1, rs2            \\
SUB.p2  rd, rs1, rs2            \\
SUB.p4  rd, rs1, rs2            \\
SUB.p8  rd, rs1, rs2            \\
MUL.p1  rd, rs1, rs2            \\
MUL.p2  rd, rs1, rs2            \\
MUL.p4  rd, rs1, rs2            \\
MUL.p8  rd, rs1, rs2            \\
SLL.p1  rd, rs1, shamt          \\
SLL.p2  rd, rs1, shamt          \\
SLL.p4  rd, rs1, shamt          \\
SLL.p8  rd, rs1, shamt          \\
SRL.p1  rd, rs1, shamt          \\
SRL.p2  rd, rs1, shamt          \\
SRL.p4  rd, rs1, shamt          \\
SRL.p8  rd, rs1, shamt          \\
ROT.p1  rd, rs1, shamt          \\
ROT.p2  rd, rs1, shamt          \\
ROT.p4  rd, rs1, shamt          \\
ROT.p8  rd, rs1, shamt          \\
MVCOP   rd, rs1                 \\
MVGPR   rd, rs1                 \\
XORT    rd, rs1, rs2            \\
ANDT    rd, rs1, rs2            \\
ORT     rd, rs1, rs2            \\
TWID    rd, rs1, b0, b1, b2, b3 \\
ADD.m   TBD                     \\
SUB.m   TBD                     \\
ACC.m   TBD                     \\
MAC.m   TBD                     \\
SLL.m   TBD                     \\
SRL.m   TBD                     \\
EQU.m   TBD                     \\
LTU.m   TBD                     \\
GTU.m   TBD                     \\
LBU.cr  rd, imm(rs1)            \\
LHU.cr  rd, imm(rs1)            \\
LW.cr   rd, imm(rs1)            \\
SB.cr   rs2, imm(rs1)           \\
SH.cr   rs2, imm(rs1)           \\
SW.cr   rs2, imm(rs1)           \\
SCATTER rs, base, offsets       \\
GATHER  rd, base, offsets       \\
}

\newpage
\subsection{Instruction Encodings}

The following standard RISC-V instruction encoding types are used in this
ISE:

\begin{center}
\begin{bytefield}[endianness=big]{32}
\bitheader{0-31}               \\
  \BB{7}{func7}
& \BB{5}{rs2  }
& \BB{5}{rs1  }
& \BB{3}{func3}
& \BB{5}{rd   }
& \BB{8}{opcode}
& \BB{8}{R-type} \\

  \BB{12}{imm[11:0]}
& \BB{5}{rs1  }
& \BB{3}{func3}
& \BB{5}{rd   }
& \BB{8}{opcode}
& \BB{8}{I-type} \\
  
  \BB{7}{imm[11:5]}
& \BB{5}{rs1  }
& \BB{5}{rs2  }
& \BB{3}{func3}
& \BB{5}{imm[4:0]}
& \BB{8}{opcode}
& \BB{8}{S-type} \\
\end{bytefield}
\end{center}

The tables below shows the encodings for all instructions in the ISE

\begin{center}
\begin{bytefield}[endianness=big]{32}
\bitheader{0-31}               \\
\BH{12}{imm[11:0]}& \BH{5}{rs1  }& \BH{3}{func3}& \BH{5}{rd   }& \BH{8}{opcode}& \BH{8}{I-type} \\
\BB{12}{imm[11:0]}& \BB{5}{rs1  }& \BB{3}{ 100 }& \BB{5}{rd   }& \BB{8}{\encopcode}& \BB{8}{LBU.cr} \\
\BB{12}{imm[11:0]}& \BB{5}{rs1  }& \BB{3}{ 101 }& \BB{5}{rd   }& \BB{8}{\encopcode}& \BB{8}{LHU.cr} \\
\BB{12}{imm[11:0]}& \BB{5}{rs1  }& \BB{3}{ 110 }& \BB{5}{rd   }& \BB{8}{\encopcode}& \BB{8}{LW.cr } \\

\BB{4}{0000}& 
\BB{2}{b0}&
\BB{2}{b1}&
\BB{2}{b2}&
\BB{2}{b3}&
\BB{5}{rs1  }& \BB{3}{ 111 }& \BB{5}{rd   }& \BB{8}{\encopcode}& \BB{8}{TWID  } \\
\end{bytefield}
  
\begin{bytefield}[endianness=big]{32}
\bitheader{0-31}               \\
\BH{7}{imm[11:5]}& \BH{5}{rs1  }& \BH{5}{rs2  }& \BH{3}{func3}& \BH{5}{imm[4:0]}& \BH{8}{opcode}& \BH{8}{S-type} \\
\BB{7}{imm[11:5]}& \BB{5}{rs1  }& \BB{5}{rs2  }& \BB{3}{ 001 }& \BB{5}{imm[4:0]}& \BB{8}{\encopcode}& \BB{8}{SB.cr} \\
\BB{7}{imm[11:5]}& \BB{5}{rs1  }& \BB{5}{rs2  }& \BB{3}{ 010 }& \BB{5}{imm[4:0]}& \BB{8}{\encopcode}& \BB{8}{SH.cr} \\
\BB{7}{imm[11:5]}& \BB{5}{rs1  }& \BB{5}{rs2  }& \BB{3}{ 011 }& \BB{5}{imm[4:0]}& \BB{8}{\encopcode}& \BB{8}{SW.cr} \\
\end{bytefield}

\begin{bytefield}[endianness=big]{32}
\bitheader{0-31}               \\
\BH{7}{func7}& \BH{5}{rs2  }& \BH{5}{rs1  }& \BH{3}{func3}& \BH{5}{rd   }& \BH{8}{opcode}& \BH{8}{R-type} \\
\BB{7}{ 0000000 }& \BB{5}{ rs2 }& \BB{5}{rs1}& \BB{3}{ 000 }& \BB{5}{rd}& \BB{8}{\encopcode}& \BB{8}{ ADD.p1 } \\
\BB{7}{ 0000001 }& \BB{5}{ rs2 }& \BB{5}{rs1}& \BB{3}{ 000 }& \BB{5}{rd}& \BB{8}{\encopcode}& \BB{8}{ ADD.p2 } \\
\BB{7}{ 0000010 }& \BB{5}{ rs2 }& \BB{5}{rs1}& \BB{3}{ 000 }& \BB{5}{rd}& \BB{8}{\encopcode}& \BB{8}{ ADD.p4 } \\
\BB{7}{ 0000011 }& \BB{5}{ rs2 }& \BB{5}{rs1}& \BB{3}{ 000 }& \BB{5}{rd}& \BB{8}{\encopcode}& \BB{8}{ ADD.p8 } \\
\BB{7}{ 0000100 }& \BB{5}{ rs2 }& \BB{5}{rs1}& \BB{3}{ 000 }& \BB{5}{rd}& \BB{8}{\encopcode}& \BB{8}{ SUB.p1 } \\
\BB{7}{ 0000101 }& \BB{5}{ rs2 }& \BB{5}{rs1}& \BB{3}{ 000 }& \BB{5}{rd}& \BB{8}{\encopcode}& \BB{8}{ SUB.p2 } \\
\BB{7}{ 0000110 }& \BB{5}{ rs2 }& \BB{5}{rs1}& \BB{3}{ 000 }& \BB{5}{rd}& \BB{8}{\encopcode}& \BB{8}{ SUB.p4 } \\
\BB{7}{ 0000111 }& \BB{5}{ rs2 }& \BB{5}{rs1}& \BB{3}{ 000 }& \BB{5}{rd}& \BB{8}{\encopcode}& \BB{8}{ SUB.p8 } \\
\BB{7}{ 0001000 }& \BB{5}{ rs2 }& \BB{5}{rs1}& \BB{3}{ 000 }& \BB{5}{rd}& \BB{8}{\encopcode}& \BB{8}{ MUL.p1 } \\
\BB{7}{ 0001001 }& \BB{5}{ rs2 }& \BB{5}{rs1}& \BB{3}{ 000 }& \BB{5}{rd}& \BB{8}{\encopcode}& \BB{8}{ MUL.p2 } \\
\BB{7}{ 0001010 }& \BB{5}{ rs2 }& \BB{5}{rs1}& \BB{3}{ 000 }& \BB{5}{rd}& \BB{8}{\encopcode}& \BB{8}{ MUL.p4 } \\
\BB{7}{ 0001011 }& \BB{5}{ rs2 }& \BB{5}{rs1}& \BB{3}{ 000 }& \BB{5}{rd}& \BB{8}{\encopcode}& \BB{8}{ MUL.p8 } \\
\BB{7}{ 0001100 }& \BB{5}{SHAMT}& \BB{5}{rs1}& \BB{3}{ 000 }& \BB{5}{rd}& \BB{8}{\encopcode}& \BB{8}{ SLL.p1 } \\
\BB{7}{ 0001101 }& \BB{5}{SHAMT}& \BB{5}{rs1}& \BB{3}{ 000 }& \BB{5}{rd}& \BB{8}{\encopcode}& \BB{8}{ SLL.p2 } \\
\BB{7}{ 0001110 }& \BB{5}{SHAMT}& \BB{5}{rs1}& \BB{3}{ 000 }& \BB{5}{rd}& \BB{8}{\encopcode}& \BB{8}{ SLL.p4 } \\
\BB{7}{ 0001111 }& \BB{5}{SHAMT}& \BB{5}{rs1}& \BB{3}{ 000 }& \BB{5}{rd}& \BB{8}{\encopcode}& \BB{8}{ SLL.p8 } \\
\BB{7}{ 0010000 }& \BB{5}{SHAMT}& \BB{5}{rs1}& \BB{3}{ 000 }& \BB{5}{rd}& \BB{8}{\encopcode}& \BB{8}{ SRL.p1 } \\
\BB{7}{ 0010001 }& \BB{5}{SHAMT}& \BB{5}{rs1}& \BB{3}{ 000 }& \BB{5}{rd}& \BB{8}{\encopcode}& \BB{8}{ SRL.p2 } \\
\BB{7}{ 0010010 }& \BB{5}{SHAMT}& \BB{5}{rs1}& \BB{3}{ 000 }& \BB{5}{rd}& \BB{8}{\encopcode}& \BB{8}{ SRL.p4 } \\
\BB{7}{ 0010011 }& \BB{5}{SHAMT}& \BB{5}{rs1}& \BB{3}{ 000 }& \BB{5}{rd}& \BB{8}{\encopcode}& \BB{8}{ SRL.p8 } \\
\BB{7}{ 0010100 }& \BB{5}{SHAMT}& \BB{5}{rs1}& \BB{3}{ 000 }& \BB{5}{rd}& \BB{8}{\encopcode}& \BB{8}{ ROT.p1 } \\
\BB{7}{ 0010101 }& \BB{5}{SHAMT}& \BB{5}{rs1}& \BB{3}{ 000 }& \BB{5}{rd}& \BB{8}{\encopcode}& \BB{8}{ ROT.p2 } \\
\BB{7}{ 0010110 }& \BB{5}{SHAMT}& \BB{5}{rs1}& \BB{3}{ 000 }& \BB{5}{rd}& \BB{8}{\encopcode}& \BB{8}{ ROT.p4 } \\
\BB{7}{ 0010111 }& \BB{5}{SHAMT}& \BB{5}{rs1}& \BB{3}{ 000 }& \BB{5}{rd}& \BB{8}{\encopcode}& \BB{8}{ ROT.p8 } \\
\BB{7}{ 0011000 }& \BB{5}{ rs2 }& \BB{5}{rs1}& \BB{3}{ 000 }& \BB{5}{rd}& \BB{8}{\encopcode}& \BB{8}{ XORT   } \\
\BB{7}{ 0011001 }& \BB{5}{ rs2 }& \BB{5}{rs1}& \BB{3}{ 000 }& \BB{5}{rd}& \BB{8}{\encopcode}& \BB{8}{ ANDT   } \\
\BB{7}{ 0011010 }& \BB{5}{ rs2 }& \BB{5}{rs1}& \BB{3}{ 000 }& \BB{5}{rd}& \BB{8}{\encopcode}& \BB{8}{ ORT    } \\
\BB{7}{ 0011011 }& \BB{5}{00000}& \BB{5}{rs1}& \BB{3}{ 000 }& \BB{5}{rd}& \BB{8}{\encopcode}& \BB{8}{ MVCOP  } \\
\BB{7}{ 0011011 }& \BB{5}{00001}& \BB{5}{rs1}& \BB{3}{ 000 }& \BB{5}{rd}& \BB{8}{\encopcode}& \BB{8}{ MVGPR  } \\
\BB{7}{ 0011100 }& \BB{5}{offsets}& \BB{5}{base}& \BB{3}{ 000 }& \BB{5}{rd}& \BB{8}{\encopcode}& \BB{8}{GATHER} \\
\BB{7}{ 0011101 }& \BB{5}{offsets}& \BB{5}{base}& \BB{3}{ 000 }& \BB{5}{rd}& \BB{8}{\encopcode}& \BB{8}{SCATTER} \\
\end{bytefield}
\end{center}
