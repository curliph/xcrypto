
All instructions occupy the \encspace encoding space of the RISC-V opcode
map, using the associated \encopcode prefix for bits $6:0$ of each 
instruction.

% ------------------------------------------------------------------------
\subsection{Move Instructions}
\label{sec:move-instructions}

These instructions are used to either move words between the RISC-V $\GPR$s
and the ISE $\CPR$s, or to conditionally move words within the $\CPR$
register file.

\instdescx{MV2GPR}
{\iencmvtwogpr}
{
    Move a $\CPR$ register to a $\GPR$ register.
}
{
\begin{itemize}
\item \INST{MV2GPR rd, crs1}{
    $\GPR[*][\OPER{rd}] \ASN \CPR[*][\OPER{crs1}]$
}
\end{itemize}
}{
    }

\instdescx{MV2COP}
{\iencmvtwocop}
{
The complement of \ASM{MV2GPR}.
Move a $\GPR$ register to a $\CPR$ register.
}{
\begin{itemize}
\item \INST{MV2COP crd, rs1}{
    $\CPR[*][\OPER{crd}] \ASN \GPR[*][\OPER{rs1}]$
}
\end{itemize}
}{
}

\instdescx{CMOV.cr, CMOVN.cr}
{\ienccmovcr\ienccmovncr}
{
These instructions implement conditional moves, and are used to implement
constant-time instruction sequences.
If the value in the $\CPR$ register addressed by {\tt rs2} is
zero / non-zero, then the value in register {\tt rs1} is moved
into register {\tt rd}.
}{
\begin{itemize}
\item \INST{CMOV.cr crd, crs1, crs2}{
    \If{$ \CPR[*][crs2] == 0$ }{
    $\CPR[*][crd] \ASN  \CPR[*][crs1]$
    }
}
\item \INST{CMOVN.cr crd, crs1, crs2}{
    \If{$ \CPR[*][crs2] != 0$ }{
    $\CPR[*][crd] \ASN  \CPR[*][crs1]$
    }
}
\end{itemize}
}{
}


% ------------------------------------------------------------------------
\subsection{Randomness}

Randomness is crucial to many parts of cryptography. Here we define an 
{\em interface} to a random source, which can be seeded and sampled using
the \ASM{RSEED.cr} and \ASM{RSAMP.cr} instructions respectivley.

We leave the exact random number generator function to the implementor,
but require that the algorithm used is cryptographically appropriate.
\designnote{
    The randomness function was left implementation specific in order to
    cater for the wide range of implementations this ISE hopes to
    enable. Different implementation platforms (FPGA/ASIC) and logic styles
    have a bearing on which sorts of RNG or TRNG make sense to use.
    It is still essential that the RNG function chosen is cryptographically
    secure.
}

\instdescx{RSEED.cr}
{\iencrseedcr}
{
\ASM{RSEED.cr} writes a 32-bit register value into the RNG, which should
then act as a seed value.
}{
\begin{itemize}
\item \INST{RSEED.cr crs1}{
    $RNG \ASN \CPR[*][crs1]$
}
\end{itemize}
}{}

\instdescx{RSAMP.cr}
{\iencrsampcr}
{
\ASM{RSAMP.cr} reads a random 32-bit value from the RNG into a register.
}{
\begin{itemize}
\item \INST{RSAMP.cr crd}{
    $\CPR[*][crd] \ASN RNG$
}
\end{itemize}
}{}


% ------------------------------------------------------------------------
\subsection{Packed Arithmetic}
\label{sec:packed-arithmetic-instructions}

The packed arithmetic instructions are designed to provide scalable, sub-word
packed operations on multiple independent values. They operate on words, and
chunk them into 1, 2, 4, 8, or 16 separate values as shown in the table
below.

These instructions are designed to allow fast and efficient implementations
of the highly data parallel operations found in cryptographic block ciphers.

The general operation of each instruction is shown below, with $f(x,y)$
used to represent the specific functionality of each instruction.

\medskip

\begin{algorithm}
\TitleOfAlgo{PackedArith}
    \KwIn{$crs1$, $crs2$ - Source registers 1 and 2}
    \KwIn{$crd$ - destination register}
    \KwIn{$pw$ - Pack width, the bit width of each operation}
    \KwIn{$f$  - The function to perform on each pair of packed operands}
    $b   \ASN 0 $\;
\While {$b < {\bf W}$}{
$\CPR[*][crd][b+pw:b] \ASN f(\CPR[*][crs1][b+pw:b],\CPR[*][crs2][b+pw:b])$\;
$b \ASN b + pw$\;
    }
\caption{Algorithm for performing packed arithmetic on words in the 
$\CPR$ register file. Here, ${\bf W}$ is the word-width in bits of the
registers, assumed to be $32$. For example, if $pw$ is 32, then the loop
will execute once, applying the function to the full 32-bit word. If $pw$
were $8$, then the function would be applied $8$ times, once to each
corresponding pair of nibbles in each word.}
\end{algorithm}

\begin{algorithm}
\TitleOfAlgo{PackedShift}
    \KwIn{$crs1$ - Source register 1}
    \KwIn{$shamt$ - Amount to shift/rotate by}
    \KwIn{$crd$ - destination register}
    \KwIn{$pw$ - Pack width, the bit width of each operation}
    \KwIn{$f$  - The function to perform on each pair of packed operands}
    $b   \ASN 0 $\;
\While {$b < {\bf W}$}{
$\CPR[*][crd][b+pw:b] \ASN f(\CPR[*][crs1][b+pw:b],shamt)$\;
$b \ASN b + pw$\;
    }
\caption{Algorithm for performing packed shifts and rotates on words in the 
$\CPR$ register file.}
\end{algorithm}

When coding these instructions as assembly mnemonics, the {\tt x} in the
instruction names below should be replaced with the pack width. Hence
for pack width 4, \ASM{SLLI.px} becomes \ASM{SLL.p4}.

\instdescx{ADD.px}
{\iencaddpx}
{
The instruction \ASM{ADD.px} treats the source and destination registers as
1, 2, 4, 8 or 16 unsigned packed fields, and adds corresponding fields
together ignoring any carry bits.

No overflow conditions are tested for, and \ASM{ADD.px} raises no exceptions.
}{
\begin{itemize}
\item \INST{ADD.px pw, crd, crs1, crs2}{
    $PackedArith(crs1,crs2,crd,pw, +)$
}
\end{itemize}
}{}


\instdescx{SUB.px}
{\iencsubpx}
{
The instruction \ASM{SUB.px} treats the source and destination registers as
1, 2, 4, 8 or 16 unsigned, packed fields, and subtracts the {\tt rs2} fields
from the corresponding {\tt rs1} fields.

No overflow conditions are tested for, and \ASM{SUB.px} raises no exceptions.
}{
\begin{itemize}
\item \INST{SUB.px pw, crd, crs1, crs2}{
    $PackedArith(crs1,crs2,crd,pw, -)$
}
\end{itemize}
}{}


\instdescx{MUL.px}
{\iencmulpx}
{
The instruction \ASM{MUL.px} treats the source and destination registers as
1, 2, 4, 8 or 16 unsigned, packed fields, and multiplies the {\tt rs2} fields
by the corresponding {\tt rs1} fields.

No overflow conditions are tested for, and \ASM{MUL.px} raises no exceptions.
}{
\begin{itemize}
\item \INST{MUL.px pw, crd, crs1, crs2}{
    $PackedArith(crs1,crs2,crd,pw, *)$
}
\end{itemize}
}{}


\instdescx{SLL[I].px}
{\iencsllpx\iencsllipx}
{
The instruction \ASM{SLL[I].px} treats the source and destination registers as
1, 2, 4, 8 or 16 unsigned, packed fields, and shifts the {\tt rs1} fields
left by the value in {\tt CSHAMT} (for \ASM{SLLI.px}) or {\tt rs2} 
(for \ASM{SLL.px}).

If the distance to shift by is greater than or equal to the size of each
packed field, then the result of the \ASM{SLL[I].px} instruction is
{\em implementation dependent}.
}{
\begin{itemize}
\item \INST{SLL.px pw, crd, crs1, crs2}{
    $PackedShift(crs1,\CPR[*][crs2],crd,pw, <<)$
}
\item \INST{SLLI.px pw, crd, crs1, cshamt}{
    $PackedShift(crs1,cshamt,crd,pw, <<)$
}
\end{itemize}
}{}


\instdescx{SRL[I].px}
{\iencsrlpx\iencsrlipx}
{
The instruction \ASM{SRL[I].px} treats the source and destination registers as
1, 2, 4, 8 or 16 unsigned, packed fields, and shifts the {\tt rs1} fields
right by the value in {\tt CSHAMT} (for \ASM{SRLI.px}) or {\tt rs2} 
(for \ASM{SRL.px}).

If the distance to shift by is greater than or equal to the size of each
packed field, then the result of the \ASM{SRL[I].px} instruction is
{\em implementation dependent}.
}{
\begin{itemize}
\item \INST{SRL.px pw, crd, crs1, crs2}{
    $PackedShift(crs1,\CPR[*][crs2],crd,pw, >>)$
}
\item \INST{SRLI.px pw, crd, crs1, cshamt}{
    $PackedShift(crs1,cshamt,crd,pw, >>)$
}
\end{itemize}
}{}


\instdescx{ROT[I].px}
{\iencrotpx\iencrotipx}
{
The instruction \ASM{ROT[I].px} treats the source and destination registers as
1, 2, 4, 8 or 16 unsigned, packed fields, and rotates the {\tt rs1} fields
left by the value in {\tt CSHAMT} (for \ASM{ROTI.px}) or {\tt rs2} 
(for \ASM{ROT.px}).
}{
\begin{itemize}
\item \INST{ROT.px pw, crd, crs1, crs2}{
    $PackedShift(crs1,\CPR[*][crs2],crd,pw, <<<)$
}
\item \INST{ROTI.px pw, crd, crs1, cshamt}{
    $PackedShift(crs1,cshamt,crd,pw, <<<)$
}
\end{itemize}
}{}

% ------------------------------------------------------------------------
\subsection{Packed Field Permutation}
\label{sec:packed-field-instructions}

These instructions allow permutation of different packed field widths.

Twiddle can be used to broadcast and re-order bit-pairs, nibbles, bytes
and half-words within words.

\instdescx{TWID.B}
{\ienctwidb}
{
The \ASM{TWID.B} instruction is used to twiddle {\em bytes within words.}
It arbitrarily maps the bytes of a source word onto the bytes of a
destination word.
This instruction doubles as a halfword permutation instruction.
}{
\begin{itemize}
\item \INST{TWID.B crd, crs1, b0, b1, b2, b3}{
    $bytes \ASN \LIST{\CPR[*][crs1][31:24],
                      \CPR[*][crs1][23:16],
                      \CPR[*][crs1][15: 8],
                      \CPR[*][crs1][ 7: 0]}$\\
    $ \CPR[*][\OPER{crd}] \ASN bytes[\OPER{b3}] \CONS
                               bytes[\OPER{b2}] \CONS
                               bytes[\OPER{b1}] \CONS
                               bytes[\OPER{b0}] $ 
}
\end{itemize}
}{}

\instdescx{TWID.Nx}
{\ienctwidnzero\ienctwidnone}
{
The \ASM{TWID.Nx} instruction is used to twiddle {\em nibbles within 
half-words}.
It works on the {\tt x}'th halfword of a register word where
$x\in{0,1}.$.
}{
\begin{itemize}
\item \INST{TWID.N0 crd, crs1, b0, b1, b2, b3}{
$ \CPR[*][\OPER{crd}][15: 0] \ASN nibbles[\OPER{b3}] \CONS
                               nibbles[\OPER{b2}] \CONS
                               nibbles[\OPER{b1}] \CONS
                               nibbles[\OPER{b0}] $ \\
$ \CPR[*][\OPER{crd}][31:16] \ASN \CPR[*][\OPER{crs1}][31:16]$ \\
}
\item \INST{TWID.N1 crd, crs1, b0, b1, b2, b3}{
$ \CPR[*][\OPER{crd}][15: 0] \ASN \CPR[*][\OPER{crs1}][15: 0]$ \\
$ \CPR[*][\OPER{crd}][31:16] \ASN nibbles[\OPER{b3}] \CONS
                               nibbles[\OPER{b2}] \CONS
                               nibbles[\OPER{b1}] \CONS
                               nibbles[\OPER{b0}] $ \\
}
\end{itemize}
}{}


\instdescx{TWID.Cx}
{\ienctwidczero\ienctwidcone\ienctwidctwo\ienctwidcthree}
{
The \ASM{TWID.[Cx]} instruction is used to twiddle {\em pairs of bits
\footnote{A {\em crumb}, if you will.}
within bytes}.
It works on the {\tt x}'th byte of a register word where
$x\in{0,1,2,3}.$.
}{
\begin{itemize}
\item \INST{TWID.C0 crd, crs1, b0, b1, b2, b3}{
$ \CPR[*][\OPER{crd}][ 7: 0] \ASN crumbs[\OPER{b3}] \CONS
                                  crumbs[\OPER{b2}] \CONS
                                  crumbs[\OPER{b1}] \CONS
                                  crumbs[\OPER{b0}] $ \\
$ \CPR[*][\OPER{crd}][31: 8] \ASN \CPR[*][\OPER{crs1}][31: 8]$\\
}
\item \INST{TWID.C1 crd, crs1, b0, b1, b2, b3}{
$ \CPR[*][\OPER{crd}][ 7: 0] \ASN \CPR[*][\OPER{crs1}][ 7: 0]$\\
$ \CPR[*][\OPER{crd}][15: 8] \ASN crumbs[\OPER{b3}] \CONS
                                  crumbs[\OPER{b2}] \CONS
                                  crumbs[\OPER{b1}] \CONS
                                  crumbs[\OPER{b0}] $ \\
$ \CPR[*][\OPER{crd}][31:16] \ASN \CPR[*][\OPER{crs1}][31:16]$\\
}
\item \INST{TWID.C2 crd, crs1, b0, b1, b2, b3}{
$ \CPR[*][\OPER{crd}][15: 0] \ASN \CPR[*][\OPER{crs1}][15: 0]$\\
$ \CPR[*][\OPER{crd}][23:16] \ASN crumbs[\OPER{b3}] \CONS
                                  crumbs[\OPER{b2}] \CONS
                                  crumbs[\OPER{b1}] \CONS
                                  crumbs[\OPER{b0}] $ \\
$ \CPR[*][\OPER{crd}][31:24] \ASN \CPR[*][\OPER{crs1}][31:24]$\\
}
\item \INST{TWID.C3 crd, crs1, b0, b1, b2, b3}{
$ \CPR[*][\OPER{crd}][23: 0] \ASN \CPR[*][\OPER{crs1}][23: 0]$\\
$ \CPR[*][\OPER{crd}][31:24] \ASN crumbs[\OPER{b3}] \CONS
                                  crumbs[\OPER{b2}] \CONS
                                  crumbs[\OPER{b1}] \CONS
                                  crumbs[\OPER{b0}] $ \\
}
\end{itemize}
}{}

% ------------------------------------------------------------------------
\subsection{Bitwise Instructions}
\label{sec:bitwise-instructions}

These instructions are designed to operate on bits within words, or
create masks which make operating on bits within words easy.

\instdescx{LUI.cr}
{\iencluicr}
{
The \ASM{LUI.cr} instruction loads a 16 bit immediate into the
upper half of a $\CPR$ register.
\designnote{The RISC-V version of \ASM{LUI} is both for constant creation
(in tandem with \ASM{ORI} and loading addresses.
Since the zeroing of the low 12 bits by \ASM{LUI} is mainly useful 
for addresses, we remove this functionality in favour of leaving the
low half of the register unmodified.
The rationale being that two instructions can still create and 32-bit
immediate, but that if we know a register is zero to begin
with, we can then use only one instruction to create masks which fit
into only a single halfword.}
The high 11 bits of the 16-bit immediate are stored in the {\tt imm11}
field of the instruction encoding. The remaining 5 bits are stored
in the {\tt imm5} field.
}{
\begin{itemize}
\item \INST{LUI.cr crd, imm16 }{
    $\CPR[*][\OPER{crd}][31:16] \ASN imm16$ \\
    $\CPR[*][\OPER{crd}][15: 0] \ASN \CPR[*][\OPER{crd}][15: 0]$ \\
}
\end{itemize}
}{}


\instdescx{LLI.cr}
{\iencllicr}
{
The \ASM{LLI.cr} instruction bitwise loads a 16 bit immediate into the
lower half of a $\CPR$ register. 
The high 11 bits of the 16-bit immediate are stored in the {\tt imm11}
field of the instruction encoding. The remaining 5 bits are stored
in the {\tt imm5} field.
}{
\begin{itemize}
\item \INST{LLI.cr crd, imm16 }{
    $\CPR[*][\OPER{crd}][31:16] \ASN \CPR[*][\OPER{crd}][31:16]$ \\
    $\CPR[*][\OPER{crd}][15: 0] \ASN imm16$ \\
}
\end{itemize}
}{}

\instdescx{BOP.cr}
{\iencbopcr}
{
The BOP.cr instruction is designed to express all bitwise logic functions, and be
amenable to bit-sliced implementations of cryptographic functions.

BOP.cr takes a 4-bit immediate {\tt lut4}, and uses the concatenated
$i$'th bits of register inputs $\CPR[*][crs1]$ and  $\CPR[*][crs2]$
as the inputs, writing the output bit of the lut to the $i$'th bit of the
destination register.
\designnote{
    This instruction must be implemented with care, since a naive
    implementation will end up with a 32-port LUT, with correspondingly
    poor timing performance.
}

}{
\begin{itemize}
\item \INST{BOP.cr crd, crs1, crs2, lut4}{
    $\forall i \in \{0..31\}$
    $ \CPR[*][\OPER{crd}][i] \ASN $
    $ \OPER{lut4}[\CPR[*][crs1][i] \CONS \CPR[*][crs2][i]] $ \\
}
\end{itemize}
}{
\paragraph{Examples:}
\begin{itemize}
\item {\tt bop.cr  c1, c2, c3, 0001    // AND}
\item {\tt bop.cr  c1, c2, c3, 0111    // OR }
\item {\tt bop.cr  c1, c2, c3, 0110    // XOR}
\end{itemize}
These examples of BOP.cr show how to encode the existing
bitwise operation instructions found in the base RISC-V architecture.
}

\instdescx{EXT.cr}
{\iencextcr}
{
This instruction allows for extraction of contiguous sequences of two-bit
pairs from words.
It takes a two bit aligned start index {\tt s}, and a length
{\tt l}, and extracts bits from $2*s$ to $(2*s)+(2*l)$, depositing them in
the destination register.
}{
\begin{itemize}
\item \INST{EXT.cr rd, rs, s, l}{
    $b \ASN 2*s$\\
    $e \ASN 2*l + b$\\
    $\CPR[*][rd][e:0] \ASN \CPR[*][rs][e:b]$
}
\end{itemize}
}{}


\instdescx{INS.cr}
{\iencinscr}
{
This instruction allows for insertion of a contiguous sequences of two-bit
pairs into words.
It takes the low $2*l$ bits of $\CPR[*][rs]$ and inserts them
into $\CPR[*][rd]$, starting at bit position $2*s$.
}{
\begin{itemize}
\item \INST{INS.cr rd, rs, s, l}{
    $b \ASN 2*s$\\
    $e \ASN 2*l + b$\\
    $\CPR[*][rd][e:b] \ASN \CPR[*][rs][e:0]$
}
\end{itemize}
}{}

\instdescx{MIX.cr}
{\ienclmixcr\ienchmixcr}
{
The MIX.cr instruction allows an arbitrary bit-wise mixing of two
register values, using a mask stored in a third register.
It first rotates  the source register value in {\tt rs1} by the amount in
{\tt shamt}, and then mixes the rotated value and the source value according
to the mask value in {\tt rs2}.

\ASM{LMIX.cr} will rotate the the value in {\tt rs1} to the right by $0..15$,
while \ASM{HMIX.cr} will rotate to the right by $16..31$. 
\designnote{This decision was made simply to make the encoding space
easier to manage. Rather than adding another 5-bit immediate field, it was
more efficient to re-use the 4-bit lut field of the BOP.cr
instruction.}
}{
\begin{itemize}
\item \INST{LMIX.cr rd, rs1, rs2,shamt}{
    $t_0 \ASN \CPR[*][rs1] \RRT shamt $\\
    $\forall i \in {0..31}$
    $    \CPR[*][rd][i] \ASN \CPR[*][rs2][i] ? t_0[i] : \CPR[*][rd][i]$
}
\item \INST{HMIX.cr rd, rs1, rs2,shamt}{
    $t_0 \ASN \CPR[*][rs1] \RRT (16 + shamt) $\\
    $\forall i \in {0..31}$
    $    \CPR[*][rd][i] \ASN \CPR[*][rs2][i] ? t_0[i] : \CPR[*][rd][i]$
}
\end{itemize}
}{
\paragraph{Examples:}
\ASM{MIX.cr} allows us to create any mix of bits from two registers
using a single 3-instruction sequence:

\begin{itemize}
\item{\tt LUI.cr   c0, 0xF0F0     // Load a mask into c0 which will select}
\item{\tt LLI.cr   c0, 0xF0F0     // alternate nibbles from each register}
\item{\tt LMIX.cr  c2, c1, c0, 0  // Odd nibbles from c1, even nibbles from c2.}
\item{\tt LMIX.cr  c3, c1, c0, 4  // Even nibbles from c1, even nibbles from c3.}
\item{\tt LMIX.cr  c4, c1, c0, 4  // Odd  nibbles from c1, even nibbles from c3.}
\end{itemize}
}

% ------------------------------------------------------------------------
\subsection{Multi-precision Arithmetic}
\label{sec:multi-precision-instructions}

Public key cryptography relies heavily on modular arithmetic over very
large integers. The following instructions are designed to allow
operations over arbitrarily large integers, using software based carry
and overflow mechanisms. This mirrors how the base RISC-V ISA currently
supports carry and overflow.

These instructions use two destination registers and between three
and four source registers. Hence, some restrictions are placed on how
registers can be addressed.

\begin{itemize}
\item
Destination registers are referred too as contiguous, even-odd pairs.
This means any two adjacently addressed registers, where the
first register address is even (zero being considered even for this
discussion) and the second address is one more than the first can be
used as the destination for a multi-precision instructions.
\designnote{
    This scheme means the function to compute the actual source registers
    from the minimum 3-bit encoding is simply to append either {\tt 00}
    or {\tt 01} depending on the even or odd address respectively.
}

For example, $(0,1), (2,3)$ and $(10,11)$ are all valid
pairs of destination registers.
$(5,6)$ is not valid because the first
is an odd number. $(6,5)$ is not valid because $6 > 5$.
The pair $(6,9)$
is not valid because $6$ and $9$ are not adjacent.

\item
The 3-bit encoding of pairs of registers is given in table 
\ref{tab:rdm-encodings}.
\end{itemize}

\begin{table}
\centering
\begin{tabular}{lll}
\hline
Encoding & crd2 & crd1 \\ \hline
0b000      & 1    & 0    \\
0b001      & 3    & 2    \\
0b010      & 5    & 4    \\
0b011      & 7    & 8    \\
0b100      & 9    & 10   \\
0b101      & 11   & 12   \\
0b110      & 13   & 14   \\
0b111      & 15   & 16   \\ \hline
\end{tabular}
\caption{
    Table showing how the {\tt rdm} field of multi-precision instructions
    is decoded into pairs of registers.
}
\label{tab:rdm-encodings}
\end{table}

\instdescx{ADD2.mp}
{\iencaddtwomp}
{
}{
\begin{itemize}
\item \INST{ADD2.mp (crd2, crd1), crs1, crs2          }{
$t \ASN ( \CPR[*][crs1] + \CPR[*][crs2] ) $\\
$\CPR[*][crd1] \ASN t[31: 0]$\\
$\CPR[*][crd2] \ASN t[64:32]$\\
}
\end{itemize}
}{}


\instdescx{ADD3.mp}
{\iencaddthreemp}
{
}{
\begin{itemize}
\item \INST{ADD3.mp (crd2, crd1), crs1, crs2, crs3     }{
$t \ASN ( \CPR[*][crs1] + \CPR[*][crs2] ) + \CPR[*][crs3] $\\
$\CPR[*][crd1] \ASN t[31: 0]$\\
$\CPR[*][crd2] \ASN t[64:32]$\\
}
\end{itemize}
}{}


\instdescx{SUB2.mp}
{\iencsubtwomp}
{
}{
\begin{itemize}
\item \INST{SUB2.mp (crd2, crd1), crs1, crs2          }{
$t \ASN ( \CPR[*][crs1] - \CPR[*][crs2] ) $\\
$\CPR[*][crd1] \ASN t[31: 0]$\\
$\CPR[*][crd2] \ASN t[64:32]$\\
}
\end{itemize}
}{}


\instdescx{SUB3.mp}
{\iencsubthreemp}
{
}{
\begin{itemize}
\item \INST{SUB3.mp (crd2, crd1), crs1, crs2, crs3     }{
$t \ASN ( \CPR[*][crs1] - \CPR[*][crs2] ) - \CPR[*][crs3] $\\
$\CPR[*][crd1] \ASN t[31: 0]$\\
$\CPR[*][crd2] \ASN t[64:32]$\\
}
\end{itemize}
}{}


\instdescx{ACC1.mp}
{\iencacconemp}
{
}{
\begin{itemize}
\item \INST{ACC1.mp (crd2, crd1), crs1}{
$t \ASN ( \CPR[*][crd2] \CONS \CPR[*][crd1] ) + \CPR[*][crs1] $\\
$\CPR[*][crd1] \ASN t[31: 0]$\\
$\CPR[*][crd2] \ASN t[64:32]$\\
}
\end{itemize}
}{}


\instdescx{ACC2.mp}
{\iencacctwomp}
{
}{
\begin{itemize}
\item \INST{ACC2.mp (crd2, crd1), crs1, crs2}{
$t \ASN ( \CPR[*][crd2] \CONS \CPR[*][crd1] ) + \CPR[*][crs1] + \CPR[*][crs2] $\\
$\CPR[*][crd1] \ASN t[31: 0]$\\
$\CPR[*][crd2] \ASN t[64:32]$\\
}
\end{itemize}
}{}


\instdescx{MAC.mp}
{\iencmacmp}
{
}{
\begin{itemize}
\item \INST{MAC.mp (crd2, crd1), crs1, crs2, crs3}{
$t \ASN ( \CPR[*][crs1] \SOP{\cdot}{u} \CPR[*][crs2] ) + \CPR[*][crs3] $\\
$\CPR[*][crd1] \ASN t[31: 0]$\\
$\CPR[*][crd2] \ASN t[64:32]$\\
}
\end{itemize}
}{}

\instdescx{SLL.mp}
{\iencsllmp}
{
}{
\begin{itemize}
\item \INST{SLL.mp (crd2, crd1), crs1, crs2, crs3     }{
$t \ASN (\CPR[*][crs1] \CONS \CPR[*][crs2]) \LSH \CPR[*][crs3][5:0]$\\
$\CPR[*][crd2] \ASN t[63:32]$ \\
$\CPR[*][crd1] \ASN t[31:0]$
}
\end{itemize}
}{}


\instdescx{SLLI.mp}
{\iencsllimp}
{
}{
\begin{itemize}
\item \INST{SLLI.mp (crd2, rd1), crs1, crs2, shamt   }{
$t \ASN (\CPR[*][crs1] \CONS \CPR[*][crs2]) \LSH shamt$\\
$\CPR[*][crd2] \ASN t[63:32]$ \\
$\CPR[*][crd1] \ASN t[31:0]$
}
\end{itemize}
}{}


\instdescx{SRL.mp}
{\iencsrlmp}
{
}{
\begin{itemize}
\item \INST{SRL.mp (crd2, crd1), crs1, crs2, crs3     }{
$t \ASN (\CPR[*][crs1] \CONS \CPR[*][crs2]) \RSH \CPR[*][crs3][5:0]$\\
$\CPR[*][crd2] \ASN t[63:32]$ \\
$\CPR[*][crd1] \ASN t[31:0]$
}
\end{itemize}
}{}


\instdescx{SRLI.mp}
{\iencsrlimp}
{
}{
\begin{itemize}
\item \INST{SRLI.mp (crd2, crd1), crs1, crs2, shamt   }{
$t \ASN (\CPR[*][crs1] \CONS \CPR[*][crs2]) \RSH shamt$\\
$\CPR[*][crd2] \ASN t[63:32]$ \\
$\CPR[*][crd1] \ASN t[31:0]$
}
\end{itemize}
}{}


\instdescx{EQU.mp}
{\ienceqump}
{
}{
\begin{itemize}
\item \INST{EQU.mp rd,      crs1, crs2, crs3     }{
$\GPR[*][rd] \ASN ( \CPR[*][crs1] = \CPR[*][crs2]) \AND \CPR[*][crs3]  $\\
}
\end{itemize}
}{}


\instdescx{LTU.mp}
{\iencltump}
{
}{
\begin{itemize}
\item \INST{LTU.mp rd,      crs1, crs2, crs3     }{
$\GPR[*][rd] \ASN ( \CPR[*][crs1] < \CPR[*][crs2] ) \IOR ( ( \CPR[*][crs1] = \CPR[*][crs2] ) \AND \CPR[*][crs3] )$
}
\end{itemize}
}{}

\instdescx{GTU.mp}
{\iencgtump}
{
}{
\begin{itemize}
\item \INST{GTU.mp rd,      crs1, crs2, crs3     }{
$\GPR[*][rd] \ASN ( \CPR[*][crs1] > \CPR[*][crs2] ) \IOR ( ( \CPR[*][crs1] = \CPR[*][crs2] ) \AND \CPR[*][crs3] )$
}
\end{itemize}
}{}

% ------------------------------------------------------------------------
\subsection{Memory Access}
\label{sec:memory-instructions}

The ISE contains several instructions listed below which allow memory
accesses to directly address the $\CPR$ register file. This means we do
not need to first load to the $\GPR$ register file, then move data into
the $\CPR$s.

All ISE memory access instructions source their base address from the
$\GPR$ register file, but read and write data in the $\CPR$s. This
prevents address manipulation needing to be done in the $\CPR$s.
\designnote{
The decision to source the base address from the $\GPR$ set rather
than a $\CPR$ was because all of the RISC-V address calculation
instructions work perfectly well on the $\GPR$s and there is no sense
in duplicating that functionality for this ISE. This free's up
encoding space, and lets the $\CPR$ state be used for the compute
work it was designed for rather than mundane address calculation.
While care must be
taken to avoid hazards between the different register sets, we
believe this decision makes sense overall.
}

ISE load and store instructions must obey the all ordering rules and
behaviour with regard to {\tt fence} instructions set out in the
RISC-V user and privilidged manuals.

There is no requirement for the ISE memory instructions to share the same
address space as the standard ISA. One may wish to completely separate
the physical memory used to store cryptographic secrets from the rest of
the virtual / logical memory space, and the ISE does not stop implementers
from doing so. This allows for hardware root-of-trust type devices to keep
their keys in tamper-proof memory, while still letting the rest of the
program exist in standard RAM/Flash.

Generally, the memory access instructions are analogous to
the standard RISC-V variants but with the following differences:

\begin{itemize}
\item The halfword and byte variants can specify which halfword or byte of
      the register they are reading or writing.
\item There is no sign-extending equivalent of the \ASM{LHU} and \ASM{LBU}
      instructions.
\item The base address is taken from the normal RISC-V $\GPR$ set, while the
      target register is always in the $\CPR$ set.
\end{itemize}

These instructions will raise a {\em load/store address misaligned} exception
if: the final source address is not aligned to the data type the instruction
is loading {\em and} the implementation does not support misaligned accesses.
The general rule here is that the \ASM{L*.cr} instructions should behave in
the same way as their RISC-V cousins in terms of exception handling,
interrupt behaviour and atomicity rules.


\instdescx{LBU.cr}
{\ienclbucr}
{
    Load an unsigned byte into a single byte of a $\CPR$, leaving other
    bytes of the $\CPR$ un-modified.
}{
\begin{itemize}
\item \INST{LBU.cr  crd, h, b, imm(rs1)}{
    $addr \ASN  \GPR[*][rs1] + signextend(imm, 32)$ \\
    $hi \ASN h*16 + b*8 + 8 $\\
    $lo \ASN h*16 + b*8 $\\
    $\CPR[*][crd][hi:lo] \ASN \MEM[*][addr]$
}
\end{itemize}
}{}

\instdescx{LHU.cr}
{\ienclhucr}
{
Load an unsigned halfword into a halfword of a $\CPR$, leaving the other
halfword un-modified.
}{
\begin{itemize}
\item \INST{LHU.cr  crd, h, imm(rs1)}{
    $addr \ASN  \GPR[*][rs1] + signextend(imm, 32)$ \\
    $hi \ASN h*16 + 16 $\\
    $lo \ASN h*16 $\\
    $\CPR[*][crd][hi:lo] \ASN \MEM[*][addr + 1] \CONS \MEM[*][addr + 0]$ \\
}
\end{itemize}
}{}

\instdescx{LW.cr}
{\ienclwcr}
{
    Load a word from memory into a $\CPR$.
}{
\begin{itemize}
\item \INST{LW.cr  crd, imm(rs1)}{
    $addr \ASN  \GPR[*][rs1] + signextend(imm, 32)$ \\
    $\CPR[*][crd] \ASN \MEM[*][addr+3] \CONS
                       \MEM[*][addr+2] \CONS
                       \MEM[*][addr+1] \CONS
                       \MEM[*][addr+0]$
}
\end{itemize}
}{}

\instdescx{SB.cr}
{\iencsbcr}
{
    Store a particular byte of a $\CPR$ into memory.
}{
\begin{itemize}
\item \INST{SB.cr  rs2, h, b, imm(rs1)}{
    $addr \ASN  \GPR[*][rs1] + signextend(imm, 32)$ \\
    $hi \ASN h*16 + b*8 + 8 $\\
    $lo \ASN h*16 + b*8 $\\
     $\MEM[*][addr] \ASN \CPR[*][rs2][hi:lo]$
}
\end{itemize}
}{}

\instdescx{SH.cr}
{\iencshcr}
{
    Store a particular halfword of a $\CPR$ into memory.
}{
\begin{itemize}
\item \INST{SH.cr  rs2, h, imm(rs1)}{
    $addr \ASN  \GPR[*][rs1] + signextend(imm, 32)$ \\
    $hi \ASN h*16 + 16 $\\
    $lo \ASN h*16$\\
    $\MEM[*][addr + 1] \ASN \CPR[*][rs2][hi+8:lo+8]$ \\
    $\MEM[*][addr + 0] \ASN \CPR[*][rs2][hi:lo]$
}
\end{itemize}
}{}


\instdescx{SW.cr}
{\iencswcr}
{
    Store a $\CPR$ into memory.
}{
\begin{itemize}
\item \INST{SW.cr  rs2, imm(rs1)}{
    $addr \ASN  \GPR[*][rs1] + signextend(imm, 32)$ \\
    $\MEM[*][addr+3] \CONS
     \MEM[*][addr+2] \CONS
     \MEM[*][addr+1] \CONS
     \MEM[*][addr+0] \ASN \CPR[*][rd]$
}
\end{itemize}
}{}


\instdescx{SCATTER.[b,h], GATHER[.b,h]}
{\iencscatterb\iencscatterh\iencgatherb\iencgatherh}
{\label{sec:scatter-gather}
The \ASM{SCATTER} and \ASM{GATHER} instructions are designed to be used as
s-box lookup instructions. The {\tt .b} variants allow four separate lookups
per instruction into a 256 entry LUT stored in memory, while the {\tt .h}
variants allow two lookups into a 16-bit LUT stored in memory. \ASM{GATHER}
will move the looked-up values from memory into a register, while \ASM{SCATTER}
will move bytes of a register word back into the LUT in memory.
\designnote{
    While it is of-course possible to emulate the {\tt .h} variants of
    \ASM{SCATTER} and \ASM{GATHER} with the {\tt.b} variants, we felt that
    the performance and efficency savings to be had from telling the hardware
    explicitly to work on halfwords were worth the extra instructions.
}

The address $a_x$ of each byte/halfword to be loaded or stored is computed by
adding an offset $x$ from the {\tt offsets} $\CPR$ register, and adding it
to the $base$ $\GPR$ register. For the {\tt .b} variants, there are four
offsets each corresponding to the four bytes of the offsets source register
word. For the {\tt .h} variant, there are two offsets corresponding to the
upper and lower halfwords of the registers.

Both instructions use the $\GPR$s to source their base address, while the
offset indices into the LUT come from the $\CPR$ registers. This mirrors
how the ISE load and store instructions compute their addresses.

The scatter instruction requires three register file reads.
Because the reads are split across the $\GPR$ and $\CPR$ registers,
no {\em extra} ports are needed on the RISC-V register file beyond the
standard two-read-one-write set.

An implementation may perform the memory accesses in any order. The instructions
will raise address misalignment exceptions in line with how misalignment is
handled in the rest of the ISA.

If any of the memory accesses cause
a {\em load access fault, load page fault, store page fault} or
{\em store access fault} then the address of the access which caused the
exception is written to the {\tt MTVAL} csr. The value taken by the
destination register bytes should an exception occur at any point in the
instruction is {\em implementation dependent}.

Implementations may abort execution of \ASM{SCATTER} or \ASM{GATHER} part way
though in order to service an interrupt. If the instructions are aborted
early, the value taken by the destination register bytes is {\em
implementation dependent}. The implementation is responsible for ensuring
that the {\tt EPC} register is set appropriately depending on whether the
interrupt is taken during, or at the end of, the instruction.
\designnote{Implementation of these instructions should consider the impact of
shared memory systems, since it will be possible for other agents in the
system to access memory during the execution of this instruction. In systems
possessing a cache, programmers must be mindful that these instructions
can cause odd behaviour due to the non-linear access pattern the instruction
creates.}

The rationale for these instructions is two fold. They help with code density,
since at-least four instructions worth of work can now be done
in one instruction. They can also help with energy efficiency, since fewer
instructions need to be fetched and travel through an execution pipeline in
order to achieve the same goal. They are not designed to enhance performance
(in terms of cycles to complete an operation), but this may be a side-effect
of some implementations.
}
{.\\
\begin{minipage}{0.5\textwidth}
\medskip
\begin{itemize}
\item \INST{GATHER.b rd, base, offsets}{
    $ a_0 \ASN \GPR[*][base] + \CPR[*][offsets][ 7: 0]$ \\
    $ a_1 \ASN \GPR[*][base] + \CPR[*][offsets][15: 8]$ \\
    $ a_2 \ASN \GPR[*][base] + \CPR[*][offsets][23:16]$ \\
    $ a_3 \ASN \GPR[*][base] + \CPR[*][offsets][31:24]$ \\
    $\CPR[*][rd][0] \ASN \MEM[*][a_0]$ \\
    $\CPR[*][rd][1] \ASN \MEM[*][a_1]$ \\
    $\CPR[*][rd][2] \ASN \MEM[*][a_2]$ \\
    $\CPR[*][rd][3] \ASN \MEM[*][a_3]$
}
\item \INST{GATHER.h rd, base, offsets}{
    $ a_0 \ASN \GPR[*][base] + \CPR[*][offsets][15: 0]$ \\
    $ a_1 \ASN \GPR[*][base] + \CPR[*][offsets][31:16]$ \\
    $\CPR[*][rd][15: 0] \ASN \MEM[*][a_0 + 1] \CONS  \MEM[*][a_0]$ \\
    $\CPR[*][rd][31:16] \ASN \MEM[*][a_1 + 1] \CONS  \MEM[*][a_1]$
}
\end{itemize}
\end{minipage}
\begin{minipage}{0.5\textwidth}
\medskip
\begin{itemize}
\item \INST{SCATTER.b rs, base, offsets}{
    $ a_0 \ASN \GPR[*][base] + \CPR[*][offsets][ 7: 0]$ \\
    $ a_1 \ASN \GPR[*][base] + \CPR[*][offsets][15: 8]$ \\
    $ a_2 \ASN \GPR[*][base] + \CPR[*][offsets][23:16]$ \\
    $ a_3 \ASN \GPR[*][base] + \CPR[*][offsets][31:24]$ \\
    $ \MEM[*][a_0] \ASN \CPR[*][rs][0]$ \\
    $ \MEM[*][a_1] \ASN \CPR[*][rs][1]$ \\
    $ \MEM[*][a_2] \ASN \CPR[*][rs][2]$ \\
    $ \MEM[*][a_3] \ASN \CPR[*][rs][3]$
}
\item \INST{SCATTER.h rs, base, offsets}{
    $ a_0 \ASN \GPR[*][base] + \CPR[*][offsets][15: 0]$ \\
    $ a_1 \ASN \GPR[*][base] + \CPR[*][offsets][31:16]$ \\
    $ \MEM[*][a_0 + 1] \CONS  \MEM[*][a_0] \ASN \CPR[*][rs][15: 0]$ \\
    $ \MEM[*][a_1 + 1] \CONS  \MEM[*][a_1] \ASN \CPR[*][rs][31:16]$
}
\end{itemize}
\end{minipage}
}
{}


\newpage
\section{Instruction Assembly Listings}

This section lists the assembly notation for the instructions, as well as
their encodings.

{\tt
MV2GPR      rd, rs1                 \\
MV2COP      rd, rs1                 \\
TWID.B      rd, rs1, b0, b1, b2, b3 \\
TWID.HN     rd, rs1, n0, n1, n2, n3 \\
TWID.LN     rd, rs1, n0, n1, n2, n3 \\
ADD.px      rd, rs1, rs2            \\
SUB.px      rd, rs1, rs2            \\
MUL.px      rd, rs1, rs2            \\
SLLI.px     rd, rs1, shamt          \\
SRLI.px     rd, rs1, shamt          \\
ROTI.px     rd, rs1, shamt          \\
SLL.px      rd, rs1, rs2            \\
SRL.px      rd, rs1, rs2            \\
ROT.px      rd, rs1, rs2            \\
XOR.cr      rd, rs1, rs2            \\
AND.cr      rd, rs1, rs2            \\
OR.cr       rd, rs1, rs2            \\
BOP.cr      rd, rs1, rs2, lut4      \\
ADD3.mp     (rd1, rd2),  rs1, rs2, rs3 \\
ADD2.mp     (rd1, rd2),  rs1, rs2  \\
SUB3.mp     (rd1, rd2),  rs1, rs2, rs3 \\
SUB2.mp     (rd1, rd2),  rs1, rs2  \\
ACC2.mp     (rd1, rd2),  rs1, rs2  \\
ACC1.mp     (rd1, rd2),  rs1       \\
MAC.mp      (rd1, rd2),  rs1, rs2, rs3 \\
MAA.mp      (rd1, rd2),  rs1, rs2, rs3, rs4 \\
SLLI.mp     (rd1, rd2),  rs1, rs2, shamt \\
SLLR.mp     (rd1, rd2),  rs1, rs2, rs3 \\
SRLI.mp     (rd1, rd2),  rs1, rs2, shamt \\
SRLR.mp     (rd1, rd2),  rs1, rs2, rs3 \\
EQU.mp      rd1,         rs1, rs2, rs3 \\
LTU.mp      rd1,         rs1, rs2, rs3 \\
GTU.mp      rd1,         rs1, rs2, rs3 \\
LBU.cr      rd , imm(rs1)           \\
LHU.cr      rd , imm(rs1)           \\
LW.cr       rd , imm(rs1)           \\
LUI.cr      rd,  imm16              \\
LLI.cr      rd,  imm16              \\
SB.cr       rs2, imm(rs1)           \\
SH.cr       rs2, imm(rs1)           \\
SW.cr       rs2, imm(rs1)           \\
SCATTER.b   rs, base, offsets       \\
GATHER.b    rd, base, offsets       \\
SCATTER.h   rs, base, offsets       \\
GATHER.h    rd, base, offsets       \\
}

\newpage
\section{Instruction Encodings}

The following tables show the encodings for all instructions in the ISE,
grouped by their layout.

\encodingsel{
    \note{The load/store byte/halfword instructions encode which bytes
    of the destination / source register they are accessing using the
    {\tt ca}, {\tt cc} and {\tt cd} bits.
    For the load instructions, {\tt cd} specifies the byte, and {\tt cc}
    specifies the halfword. For the store instructions {\tt ca}
    specifies the byte, and {\tt cc} also specifies the halfword.
}
}{ fig:encodings-else }


\encodingspx{
Packed register type instruction encodings. 
The fields {\tt ca, cb} and {\tt cc} are used by the {\tt *.px} instructions
to encode the pack width of their operations. The three bits are
concatenated into a 3-bit vector $a,b,c$ and then decoded into pack widths
using the following lookup table. \\
\begin{itemize}
\item $ca,cb,cc = 000 \Rightarrow $ Pack width $1$  $\Rightarrow$ 1 32-bit word
\item $ca,cb,cc = 001 \Rightarrow $ Pack width $2$  $\Rightarrow$ 2 halfwords
\item $ca,cb,cc = 010 \Rightarrow $ Pack width $4$  $\Rightarrow$ 4 bytes
\item $ca,cb,cc = 011 \Rightarrow $ Pack width $8$  $\Rightarrow$ 8 nibbles
\item $ca,cb,cc = 100 \Rightarrow $ Pack width $16$ $\Rightarrow$ 16 crumbs (2-bit values)
\item $ca,cb,cc = 101 $ - Reserved. Should be treated as an invalid opcode.
\item $ca,cb,cc = 110 $ - Reserved. Should be treated as an invalid opcode.
\item $ca,cb,cc = 111 $ - Reserved. Should be treated as an invalid opcode.
\end{itemize}
}{
fig:encodings-rtype
}


\encodingsmp{
Multi-precision integer instruction encodings. All instructions in
this class are identified first by the {\tt f3} field, then by the {\tt c}
field. The {\tt c} field splits comparison operations from compute operations.
The {\tt mo} and {\tt op} fields then further identify the instructions.
Note that the {\tt rd} field for the comparison instructions is a RISC-V
$\GPR$ register address.
}{
fig:encodings-mp
}
